\documentclass[SE,lsstdraft,STR,toc]{lsstdoc}
\usepackage{geometry}
\usepackage{longtable,booktabs}
\usepackage{enumitem}
\usepackage{arydshln}

\input meta.tex

\providecommand{\tightlist}{
  \setlength{\itemsep}{0pt}\setlength{\parskip}{0pt}}

\begin{document}

\def\milestoneName{Camera Rotator Functional Re-Verification}
\def\milestoneId{LVV-P59}
\def\product{SIT-COM Integration}

\setDocCompact{true}

\title{ LVV-P59 Camera Rotator Functional Re-Verification Test Plan and Report}
\setDocRef{\lsstDocType-\lsstDocNum}
\date{\vcsdate}
\setDocUpstreamLocation{\url{https://github.com/lsst/lsst-texmf/examples}}
\author{ Kevin Siruno }

\input history_and_info.tex


\setDocAbstract{
This is the test plan and report for LVV-P59 (Camera Rotator Functional Re-Verification),
an LSST milestone pertaining to the System Engineering Subsystem.
}


\maketitle

\section{Introduction}
\label{sect:intro}


\subsection{Objectives}
\label{sect:objectives}

The objective of this test plan is to re-verify the functional
requirements of the camera rotator's hardware and software, after
shipment from the vendors facility to the Summit, as defined in \citeds{LTS-206}
and \citeds{LTS-160}. This test campaign will only exercise the functionality
that was executed previously and meets the following criteria:

\begin{itemize}
\tightlist
\item
  Only requires the camera rotator to be operable
\item
  Only requires the vendors EUI software and hardware via local control
\item
  Only requires a laser tracker
\item
  Does \textbf{NOT} require the camera rotator to be loaded with the
  camera simulated mass or actual camera hardware
\end{itemize}

The hardware functional requirements were previously verified during the
test campaign by the vendor at the vendors facility and accepted by LSST
during the Factory Acceptance Test review.



\subsection{System Overview}
\label{sect:systemoverview}

The Camera Rotator is mounted to the Camera Hexapod with the primary
function of rotating the camera about the Camera's central axis.


\subsection{Document Overview}
\label{sect:docoverview}

This document was generated from Jira, obtaining the relevant information from the 
\href{https://jira.lsstcorp.org/secure/Tests.jspa#/testPlan/LVV-P59}{LVV-P59}
~Jira Test Plan and related Test Cycles (
  \href{https://jira.lsstcorp.org/secure/Tests.jspa#/testCycle/LVV-C110}{LVV-C110}
).

Section \ref{sect:intro} provides an overview of the test campaign, the system under test (\product{}), the applicable documentation, and explains how this document is organized.
Section \ref{sect:configuration}  describes the configuration used for this test.
Section \ref{sect:personnel} describes the necessary roles and lists the individuals assigned to them.
%Section \ref{sect:plannedtestactivities} provides the list of planned test cycles and test cases, including all relevant information that fully describes the test campaign.

Section \ref{sect:overview} provides a summary of the test results, including an overview in Table \ref{table:summary}, an overall assessment statement and suggestions for possible improvements.
Section \ref{sect:detailedtestresults} provides detailed results for each step in each test case.

The current status of test plan LVV-P59 in Jira is \textbf{ Draft }.

\subsection{References}
\label{sect:references}
\renewcommand{\refname}{}
\bibliography{lsst,refs,books,refs_ads,local}
\section{Test Configuration}
\label{sect:configuration}

\subsection{Data Collection}

  Observing is not required for this test campaign.

\subsection{Verification Environment}
\label{sect:hwconf}
  The Camera Rotator will be verified in a climate controlled environment
on the 3rd floor of the Summit Facility integrated with the Camera Cable
Wrap on the Camera Cart.


  \subsection{Entry Criteria}
  In order to test the Camera Rotator functionality, the following
criteria must be met first:

\begin{itemize}
\tightlist
\item
  All the test setup for the Data Acquisition system must be completed
  and ready to record data for the laser tracker and current probes
\item
  The Laser tracker and SMR's are installed and setup
\item
  The Inductive current probes are installed and setup
\item
  All utilities and electrical connections are hooked up and allow the
  Camera Rotator to be powered on and controlled
\item
  The EFD must be set up to be able to store events and telemetry data
\end{itemize}


  \subsection{Exit Criteria}
  In order for this event to be considered complete, the following
criteria must be met:

\begin{itemize}
\tightlist
\item
  Raw test data, events, and telemetry have been saved for the Camera
  Rotator.
\item
  All test data has been analyzed and post processed.
\item
  All test steps have been statused in the Jira Test Cases within this
  Test Plan and actual results populated as required.
\item
  A summary of the results of the test campaign has been captured in the
  Overall Assessment and Recommended Improvements fields of this Test
  Plan
\item
  A link to the verification artifacts used to produce the summary of
  results has been populated in the Verification Artifacts field of this
  Test Plan
\item
  Any failures have been captured in the
  \href{https://jira.lsstcorp.org/projects/FRACAS/issues/}{FRACAS}
  project
\end{itemize}


  \subsection{PMCS Activity}
  See Epics in Traceability Tab


\newpage
\section{Personnel}
\label{sect:personnel}

The personnel involved in the test campaign is shown in the following table.

\begin{longtable}{p{3cm}p{3cm}p{3cm}p{6cm}}
\hline
\multicolumn{2}{r}{Test Plan (LVV-P59) owner:} &
\multicolumn{2}{l}{\textbf{ Kevin Siruno } }\\\hline
\multicolumn{2}{r}{ LVV-C110 owner:} &
\multicolumn{2}{l}{\textbf{
    Kevin Siruno
}
} \\\hline
\textbf{Test Case} & \textbf{Assigned to} & \textbf{Executed by} & \textbf{Additional Test Personnel} \\ \hline
\href{https://jira.lsstcorp.org/secure/Tests.jspa#/testCase/LVV-T1577}{LVV-T1577}
& {\small Kevin Siruno } & {\small  } &
\begin{minipage}[]{6cm}
\smallskip
{\small (1) Software Engineer\\
(1) Hardware Engineer
 }
\medskip
\end{minipage}
\\ \hline
\href{https://jira.lsstcorp.org/secure/Tests.jspa#/testCase/LVV-T1576}{LVV-T1576}
& {\small Kevin Siruno } & {\small  } &
\begin{minipage}[]{6cm}
\smallskip
{\small (1) Mechanical Engineer/Optical Engineer\\
(1) Electrical Engineer
 }
\medskip
\end{minipage}
\\ \hline
\end{longtable}

\newpage

\section{Overview of the Test Results}
\label{sect:overview}

\subsection{Summary}
\label{sect:summarytable}

\begin{longtable}{p{0.12\textwidth}p{0.2\textwidth}p{0.56\textwidth}p{0.12\textwidth}}
\toprule

  \multicolumn{3}{c}{ Test Cycle {\bf LVV-C110: Camera Rotator Re-verification
 }} \\\hline

  {\bf \footnotesize test case} & {\bf \footnotesize status} & {\bf \footnotesize comment} & {\bf \footnotesize issues} \\\toprule

    \href{https://jira.lsstcorp.org/secure/Tests.jspa#/testCase/LVV-T1577}{LVV-T1577}
    & Not Executed &
    \begin{minipage}[]{0.56\textwidth}
    \smallskip
    
    \medskip
    \end{minipage}
    &     \\\hline
    \href{https://jira.lsstcorp.org/secure/Tests.jspa#/testCase/LVV-T1576}{LVV-T1576}
    & Not Executed &
    \begin{minipage}[]{0.56\textwidth}
    \smallskip
    
    \medskip
    \end{minipage}
    &     \\\hline

\caption{Test Results Summary}
\label{table:summary}
\end{longtable}

\subsection{Overall Assessment}
\label{sect:overallassessment}

Not yet available.

\subsection{Recommended Improvements}
\label{sect:recommendations}

Not yet available.

\newpage
\section{Detailed Test Results}
\label{sect:detailedtestresults}


  \subsection{Test Cycle LVV-C110 }

Open test cycle {\it \href{https://jira.lsstcorp.org/secure/Tests.jspa#/testrun/LVV-C110}{Camera Rotator Re-verification
}} in Jira.

  Camera Rotator Re-verification
\\
  Status: Not Executed

  Re-verify the hardware and software requirements for the Camera rotator
that were previously tested by MOOG.


  \subsubsection{Software Version/Baseline}
    \begin{enumerate}
\tightlist
\item
  Camera Rotator Control Software with SAL v3.5
\item
  Mariadb EFD with SAL v3.5
\end{enumerate}


  \subsubsection{Configuration}
    No varying configuration between test cycles.


  \subsubsection{Test Cases in LVV-C110 Test Cycle}


    \paragraph{Test Case LVV-T1577 - Camera Rotator Software Functional Re-verification
 }\mbox{}\\

Open  \href{https://jira.lsstcorp.org/secure/Tests.jspa#/testCase/LVV-T1577}{\textit{ LVV-T1577 } }
test case in Jira.

    The objective of this test case is to re-verify the functional
requirements of the camera rotator's software, after shipment of the
hardware from the vendor's facility to the Summit, as defined in \citeds{LTS-206}
and \citeds{LTS-160}. This test case will only exercise the functionality that
was executed previously and meets the following criteria:

\begin{itemize}
\tightlist
\item
  Only requires the camera rotator to be operable
\item
  Only requires the vendors EUI software and hardware via local control
\item
  Does \textbf{NOT} require the camera rotator to be loaded with the
  camera simulated mass or actual camera hardware
\end{itemize}

The software functional requirements were previously verified during the
test campaign by the vendor at the vendor's facility and accepted by
LSST during the Factory Acceptance Test review. The test procedure used
during the vendor's acceptance testing is the \emph{LSST
Hexapods-Rotator Software Acceptance Test Procedure} which is attached
to this test case. The test steps of this test case are taken directly
from that document on how to perform the test in a similar way as was
performed previously and includes changes noted by the vendor.\\
~\\
The \emph{LSST Hexapods-Rotator Software Acceptance Test
Procedure\_Completed} has also been attached to this test case for
additional information on which tests were passed.\\
~\\
See the Confluence page \emph{Updated Middleware Test in SLAC at
10/15/2019} linked in the Traceability Tab for additional results of the
State Transition Test done by LSST using a GUI and SAL 3.10.\\
~\\
See the attached \emph{LSST Rotator Operator's Manual} for more
information on how to operate the rotator.\\
~\\


    \textbf{ Preconditions}:\\
    Prior to the execution of this test case to re-verify the Camera Rotator
software functional requirements, the following Summit tasks must be
completed:\\

\begin{itemize}
\tightlist
\item
  The cables and cabinets have been checked~

  \begin{itemize}
  \tightlist
  \item
    \url{https://jira.lsstcorp.org/browse/SUMMIT-3231}
  \end{itemize}
\item
  Boxes for the hexapod/rotator have been transported to the 3rd level

  \begin{itemize}
  \tightlist
  \item
    \url{https://jira.lsstcorp.org/browse/SUMMIT-3230}
  \end{itemize}
\item
  Test fit Camera Hexapod with Offset

  \begin{itemize}
  \tightlist
  \item
    \url{https://jira.lsstcorp.org/browse/SUMMIT-3293}
  \end{itemize}
\item
  The Hexapod and Rotator have been installed on camera cart

  \begin{itemize}
  \tightlist
  \item
    \url{https://jira.lsstcorp.org/browse/SUMMIT-3224}
  \end{itemize}
\item
  The Camera hexapod/rotator has been connected to the electronics
  cabinets and the connections have been tested

  \begin{itemize}
  \tightlist
  \item
    \url{https://jira.lsstcorp.org/browse/SUMMIT-3294}
  \end{itemize}
\end{itemize}


    Execution status: {\bf Not Executed }

    Final comment:\\


    Detailed step results:

    \begin{longtable}{p{1cm}p{2cm}p{13cm}}
    \hline
    {Step} & \multicolumn{2}{c}{Description, Results and Status}\\ \hline
      1 & Description &

      \begin{minipage}[t]{13cm}{\footnotesize
      Section 3.2.1 of the attached Software Acceptance Test Procedure\\
Test Sequence \#1 - PositionSet and Move Commands

\begin{itemize}
\tightlist
\item
  In the enabled/stationary state, send a positionSet command of 12 deg.
  Confirm that the rotator does not move.
\item
  Send a positionSet command of 15deg. Confirm that the rotator does not
  move.
\item
  Send a move command. Confirm that the rotator moves to 15 deg.
\end{itemize}

      \vspace{\dp0}
      } \end{minipage} \\
      \\ \cdashline{2-3}


      & Expected Result &

      \begin{minipage}[t]{13cm}{\footnotesize
      As initially tested in the LSST Hexapods-Rotator Software Acceptance
Test Procedure\_Completed, Sheet 8-9, the Rotator only moves when given
a positionSet command AND a move command controlled through the EUI
device.~

      \vspace{\dp0}
      } \end{minipage} \\
      \\ \cdashline{2-3}

      & \begin{minipage}[t]{2cm}{Actual\\ Result}\end{minipage}   & 
      \begin{minipage}[t]{13cm}{\footnotesize
      
      \vspace{\dp0}
      } \end{minipage} \\
      \\ \cdashline{2-3}


      & Status          & Not Executed \\ \hline

      2 & Description &

      \begin{minipage}[t]{13cm}{\footnotesize
      Section 3.2.1 of the attached Software Acceptance Test Procedure\\
Test Sequence \#2 - StopCommand

\begin{itemize}
\tightlist
\item
  In the enabled/stationary state, send a positionSet command of 50 deg.
  Confirm that the rotator does not move.
\item
  Send a move command.~
\item
  While the rotator is still moving, send a Stop command. Confirm that
  the system quickly comes to a stop before reaching the 50 deg
  position.
\item
  Send a positionSet command of 60 deg followed by a move command.
  Confirm the rotator moves to the commanded position following the
  previous stop command.
\end{itemize}

      \vspace{\dp0}
      } \end{minipage} \\
      \\ \cdashline{2-3}


      & Expected Result &

      \begin{minipage}[t]{13cm}{\footnotesize
      As initially tested in the LSST Hexapods-Rotator Software Acceptance
Test Procedure\_Completed, Sheet 9, the rotator is able to receive and
execute a stop command while in motion through the EUI device.

      \vspace{\dp0}
      } \end{minipage} \\
      \\ \cdashline{2-3}

      & \begin{minipage}[t]{2cm}{Actual\\ Result}\end{minipage}   & 
      \begin{minipage}[t]{13cm}{\footnotesize
      
      \vspace{\dp0}
      } \end{minipage} \\
      \\ \cdashline{2-3}


      & Status          & Not Executed \\ \hline

      3 & Description &

      \begin{minipage}[t]{13cm}{\footnotesize
      Section 3.2.1 of the attached Software Acceptance Test Procedure\\
Test Sequence \#3 - VelocitySet and MoveConstantVelocity Commands

\begin{itemize}
\tightlist
\item
  In the enabled/stationary state, send a velocitySet command of 0.01
  deg/s and 15 seconds. Confirm that the rotator does not move.
\item
  Send a velocitySet command of 0.02 deg/s and 10 seconds. Confirm that
  the rotator does not move.
\item
  Send a moveConstantVelocity command. Confirm that the rotator moves
  approximately 0.2 degrees total over 10 seconds before coming to a
  stop and transitioning back to enabled/stationary state.~
\end{itemize}

      \vspace{\dp0}
      } \end{minipage} \\
      \\ \cdashline{2-3}


      & Expected Result &

      \begin{minipage}[t]{13cm}{\footnotesize
      As initially tested in the LSST Hexapods-Rotator Software Acceptance
Test Procedure\_Completed, Sheet 9, the rotator is able to move 0.2
degrees over 10 seconds ~only when the moveConstantVelocity command is
given through the EUI.

      \vspace{\dp0}
      } \end{minipage} \\
      \\ \cdashline{2-3}

      & \begin{minipage}[t]{2cm}{Actual\\ Result}\end{minipage}   & 
      \begin{minipage}[t]{13cm}{\footnotesize
      
      \vspace{\dp0}
      } \end{minipage} \\
      \\ \cdashline{2-3}


      & Status          & Not Executed \\ \hline

      4 & Description &

      \begin{minipage}[t]{13cm}{\footnotesize
      Section 3.2.2 of the attached Software Acceptance Test Procedure\\
Test Sequence \#6 - configureVelocity Command

\begin{itemize}
\tightlist
\item
  In the enabled/stationary state, send a configureVelocity command of 4
  deg/s. Confirm that the command is rejected for being out of
  acceptable range.
\item
  Send a configureVelocity command of 0.5 deg/s. Confirm that this
  command is accepted.
\item
  Send a positionSet command to a position 10 deg away from the current
  position. Send a move command. Confirm that the move is completed in
  approximately 20 seconds.
\item
  Record the corresponding DDS events that were generated.
\end{itemize}

      \vspace{\dp0}
      } \end{minipage} \\
      \\ \cdashline{2-3}


      & Expected Result &

      \begin{minipage}[t]{13cm}{\footnotesize
      Using the EUI, the rotator is able to receive the configureVelocity,
positionSet and move command to ~set the velocity to 0.5 deg/s and moves
to a position 10 deg away from the current position in approximately 20
seconds.~

      \vspace{\dp0}
      } \end{minipage} \\
      \\ \cdashline{2-3}

      & \begin{minipage}[t]{2cm}{Actual\\ Result}\end{minipage}   & 
      \begin{minipage}[t]{13cm}{\footnotesize
      
      \vspace{\dp0}
      } \end{minipage} \\
      \\ \cdashline{2-3}


      & Status          & Not Executed \\ \hline

      5 & Description &

      \begin{minipage}[t]{13cm}{\footnotesize
      Section 3.2.2 of the attached Software Acceptance Test Procedure\\
Test Sequence \#7 - configureAcceleration Command

\begin{itemize}
\tightlist
\item
  In the enabled/stationary state with the configureVelocity command
  from the previous still in effect, send a configureAcceleration
  command of 2 deg/s\^{}2. Confirm that the command is rejected for
  being out of acceptable range.
\item
  Send a configureAcceleration command of 0.05 deg/s\^{}2. Confirm that
  the command is accepted.
\item
  Send a positionSet command to a position 10 deg away from the current
  position. Send a move command. Confirm that the move is completed in
  approximately 30 seconds.~
\item
  Record the corresponding DDS events that were generated.
\end{itemize}

      \vspace{\dp0}
      } \end{minipage} \\
      \\ \cdashline{2-3}


      & Expected Result &

      \begin{minipage}[t]{13cm}{\footnotesize
      Using the EUI, the rotator is able to receive a configureAcceleration,
positionSet and move command to set the acceleration to 0.05 deg/\^{}2
and moves to a position 10 deg away from the current position in
approximately 30 seconds.~

      \vspace{\dp0}
      } \end{minipage} \\
      \\ \cdashline{2-3}

      & \begin{minipage}[t]{2cm}{Actual\\ Result}\end{minipage}   & 
      \begin{minipage}[t]{13cm}{\footnotesize
      
      \vspace{\dp0}
      } \end{minipage} \\
      \\ \cdashline{2-3}


      & Status          & Not Executed \\ \hline

      6 & Description &

      \begin{minipage}[t]{13cm}{\footnotesize
      Section 3.3.1 of the attached Software Acceptance Test Procedure

\begin{itemize}
\tightlist
\item
  At startup, confirm that the system starts in the Offline/PublishOnly
  state
\item
  Send an offline substate trigger of systemReady. Confirm system goes
  into Offline/Available substate.
\item
  Send an EnterControl trigger and confirm the system transitions from
  Offline/Availale to Standby state
\item
  Send a Start trigger and confirm the system transitions from Standby
  to Disabled state.
\item
  Send an Enable trigger and confirm the system transitions from
  Disabled to Enabled state
\item
  Send a Disable trigger and confirm the system transitions from Enabled
  to Disabled state
\item
  Send a Standby trigger and confirm the system transitions from
  Disabled state to Standby state
\item
  Send a exitControl trigger and confirm the system transitions from
  Standby state to Offline state
\item
  Return to the Enabled state. Trip the safety interlock switch and
  confirm the system transitions to Fault state.
\item
  Reset the safety interlock and send a ClearError trigger. Confirm the
  system transitions from Fault state to Offline state.
\item
  Repeat all tests on both hexapods and rotator.
\end{itemize}

      \vspace{\dp0}
      } \end{minipage} \\
      \\ \cdashline{2-3}


      & Expected Result &

      \begin{minipage}[t]{13cm}{\footnotesize
      As initially tested in the LSST Hexapods-Rotator Software Acceptance
Test Procedure\_Completed, Sheet 11-12, the rotator system transitions
to the correct states as defined in LTS-160, Section 3.3.

      \vspace{\dp0}
      } \end{minipage} \\
      \\ \cdashline{2-3}

      & \begin{minipage}[t]{2cm}{Actual\\ Result}\end{minipage}   & 
      \begin{minipage}[t]{13cm}{\footnotesize
      
      \vspace{\dp0}
      } \end{minipage} \\
      \\ \cdashline{2-3}


      & Status          & Not Executed \\ \hline

      7 & Description &

      \begin{minipage}[t]{13cm}{\footnotesize
      Section 5.1 of the attached Software Acceptance Test Procedure

\begin{itemize}
\tightlist
\item
  In the Enabled state, unplug an encoder cable for one of the rotator
  motors. Confirm that a Drive Fault event is created and the system
  transitions to Fault state.
\item
  In the Enabled state, unplug a linear encoder cable for the rotator.
  Confirm that a Linear Encoder Error event is created and the system
  transitions to Fault state.
\item
  Set the Following Error Threshold parameter to a very small value
  (0.0001 deg or smaller) and command a PositionSet/Move. Confirm that a
  Following Error event is created and the system transitions to Fault
  state.~
\item
  Activate the positive software limit using a special control program.
  Confirm that a Positive Limit Switch error message is created and the
  system transitions to Fault state.
\item
  Activate the negative software limit using a special control program.
  Confirm that a Negative Limit Switch error message is created and the
  system transitions to Fault State.
\item
  Unplug the Ethercat cable between the control PC and the Copley XE2
  drive. Confirm that an Ethercat Problem event is created and the
  system transitions to Fault state.
\end{itemize}

      \vspace{\dp0}
      } \end{minipage} \\
      \\ \cdashline{2-3}


      & Expected Result &

      \begin{minipage}[t]{13cm}{\footnotesize
      As initially tested in the LSST Hexapods-Rotator Software Acceptance
Test Procedure\_Completed, Sheet 14-15, the rotator displays the events
on the EUI whenever a change in state occurs as specified in LTS-160,
Section 5.

      \vspace{\dp0}
      } \end{minipage} \\
      \\ \cdashline{2-3}

      & \begin{minipage}[t]{2cm}{Actual\\ Result}\end{minipage}   & 
      \begin{minipage}[t]{13cm}{\footnotesize
      
      \vspace{\dp0}
      } \end{minipage} \\
      \\ \cdashline{2-3}


      & Status          & Not Executed \\ \hline

    \end{longtable}


    \paragraph{Test Case LVV-T1576 - Camera Rotator Hardware Functional Re-verification
 }\mbox{}\\

Open  \href{https://jira.lsstcorp.org/secure/Tests.jspa#/testCase/LVV-T1576}{\textit{ LVV-T1576 } }
test case in Jira.

    The objective of this test case is to re-verify the functional
requirements of the camera rotator's hardware, after shipment from the
vendors facility to the Summit, as defined in \citeds{LTS-206}. This test case
will only exercise the functionality that was executed previously and
meets the following criteria:

\begin{itemize}
\tightlist
\item
  Only requires the camera rotator to be operable
\item
  Only requires the vendors EUI software and hardware via local control
\item
  Only requires a laser tracker
\item
  Does \textbf{NOT} require the camera rotator to be loaded with the
  camera simulated mass or actual camera hardware
\end{itemize}

The hardware functional requirements were previously verified during the
test campaign by the vendor at the vendors facility and accepted by LSST
during the Factory Acceptance Test review. The test procedure used
during the vendor's acceptance testing is the \emph{LSST
Hexapods-Rotator Acceptance Test Procedure} which is attached to this
test case. The test steps of this test case reference that document for
the details on how to perform the test in a similar way as was performed
previously and includes deviations to that document due to the
differences in the verification configuration and deviations to
requirements granted to the vendor by LSST.\\
~\\
The \emph{LSST Hexapods-Rotator Acceptance Test Report} has also been
attached to this test case for additional information on how the tests
were performed. The section numbering in this document matches that of
the procedure.\\
~\\
See the attached \emph{LSST Rotator Operator's Manual} for more
information on how to operate the rotator.


    \textbf{ Preconditions}:\\
    {Prior to the execution of this test case to re-verify the Camera
Rotator hardware functional requirements, the following Summit tasks
must be completed:}

\begin{itemize}
\tightlist
\item
  The cables and cabinets have been checked~

  \begin{itemize}
  \tightlist
  \item
    \url{https://jira.lsstcorp.org/browse/SUMMIT-3231}
  \end{itemize}
\item
  Boxes for the hexapod/rotator have been transported to the 3rd level

  \begin{itemize}
  \tightlist
  \item
    \url{https://jira.lsstcorp.org/browse/SUMMIT-3230}
  \end{itemize}
\item
  Test fit Camera Hexapod with Offset

  \begin{itemize}
  \tightlist
  \item
    \url{https://jira.lsstcorp.org/browse/SUMMIT-3293}
  \end{itemize}
\item
  The Hexapod and Rotator have been installed on camera cart

  \begin{itemize}
  \tightlist
  \item
    \url{https://jira.lsstcorp.org/browse/SUMMIT-3224}
  \end{itemize}
\item
  The Camera hexapod/rotator has been connected to the electronics
  cabinets and the connections have been tested

  \begin{itemize}
  \tightlist
  \item
    \url{https://jira.lsstcorp.org/browse/SUMMIT-3294}
  \end{itemize}
\item
  The Offset has been installed on the Integrator

  \begin{itemize}
  \tightlist
  \item
    \url{https://jira.lsstcorp.org/browse/SUMMIT-3281}
  \end{itemize}
\item
  The setup for the laser tracker, current probes and data acquisition
  system has been completed

  \begin{itemize}
  \tightlist
  \item
    \url{https://jira.lsstcorp.org/browse/SUMMIT-3431}
  \end{itemize}
\end{itemize}


    Execution status: {\bf Not Executed }

    Final comment:\\


    Detailed step results:

    \begin{longtable}{p{1cm}p{2cm}p{13cm}}
    \hline
    {Step} & \multicolumn{2}{c}{Description, Results and Status}\\ \hline
      1 & Description &

      \begin{minipage}[t]{13cm}{\footnotesize
      Follow Section 3.4.1 of the LSST Hexapods-Rotator Acceptance Test
Procedure, Sheet 46.

      \vspace{\dp0}
      } \end{minipage} \\
      \\ \cdashline{2-3}


      & Expected Result &

      \begin{minipage}[t]{13cm}{\footnotesize
      The axis of rotation is visually confirmed to be along the z-axis and is
consistent to the results of the initial tests conducted by the vendor
as seen in the LSST Hexapods-Rotator Acceptance Test Report, Sheet 48.

      \vspace{\dp0}
      } \end{minipage} \\
      \\ \cdashline{2-3}

      & \begin{minipage}[t]{2cm}{Actual\\ Result}\end{minipage}   & 
      \begin{minipage}[t]{13cm}{\footnotesize
      
      \vspace{\dp0}
      } \end{minipage} \\
      \\ \cdashline{2-3}


      & Status          & Not Executed \\ \hline

      2 & Description &

      \begin{minipage}[t]{13cm}{\footnotesize
      Follow Section 3.4.2 of the LSST Hexapods-Rotator Acceptance Test
Procedure, Sheet 46 using a laser tracker.

      \vspace{\dp0}
      } \end{minipage} \\
      \\ \cdashline{2-3}


      & Expected Result &

      \begin{minipage}[t]{13cm}{\footnotesize
      The Camera Rotator is tested every 30 degrees across the entire range of
motion and the maximum angle error is found to be less than 0.009
degrees.

      \vspace{\dp0}
      } \end{minipage} \\
      \\ \cdashline{2-3}

      & \begin{minipage}[t]{2cm}{Actual\\ Result}\end{minipage}   & 
      \begin{minipage}[t]{13cm}{\footnotesize
      
      \vspace{\dp0}
      } \end{minipage} \\
      \\ \cdashline{2-3}


      & Status          & Not Executed \\ \hline

      3 & Description &

      \begin{minipage}[t]{13cm}{\footnotesize
      Follow Section 3.4.3 of the LSST Hexapods-Rotator Acceptance Test
Procedure, Sheet 47.\\
\emph{Might Delete this step.}

      \vspace{\dp0}
      } \end{minipage} \\
      \\ \cdashline{2-3}


      & Expected Result &

      \begin{minipage}[t]{13cm}{\footnotesize
      The encoder values are verified to be correct by comparing the values
recorded using the laser tracker.

      \vspace{\dp0}
      } \end{minipage} \\
      \\ \cdashline{2-3}

      & \begin{minipage}[t]{2cm}{Actual\\ Result}\end{minipage}   & 
      \begin{minipage}[t]{13cm}{\footnotesize
      
      \vspace{\dp0}
      } \end{minipage} \\
      \\ \cdashline{2-3}


      & Status          & Not Executed \\ \hline

      4 & Description &

      \begin{minipage}[t]{13cm}{\footnotesize
      Follow Section 3.4.4 of the LSST Hexapods-Rotator Acceptance Test
Procedure, Sheet 47.

      \vspace{\dp0}
      } \end{minipage} \\
      \\ \cdashline{2-3}


      & Expected Result &

      \begin{minipage}[t]{13cm}{\footnotesize
      Without damaging the system, the hard end-stops are verified to allow
the Camera Rotator to meet the 180 operational range without going over
2 degrees past the limit.

      \vspace{\dp0}
      } \end{minipage} \\
      \\ \cdashline{2-3}

      & \begin{minipage}[t]{2cm}{Actual\\ Result}\end{minipage}   & 
      \begin{minipage}[t]{13cm}{\footnotesize
      
      \vspace{\dp0}
      } \end{minipage} \\
      \\ \cdashline{2-3}


      & Status          & Not Executed \\ \hline

      5 & Description &

      \begin{minipage}[t]{13cm}{\footnotesize
      Follow Section 3.4.5.1 of the LSST Hexapods-Rotator Acceptance Test
Procedure, Sheet 48.

      \vspace{\dp0}
      } \end{minipage} \\
      \\ \cdashline{2-3}


      & Expected Result &

      \begin{minipage}[t]{13cm}{\footnotesize
      The Camera Rotator is able to reach the required velocity of 3.5 deg/s
as verified before per the LSST Hexapods-Rotator Acceptance Test Report,
Sheet 50-52

      \vspace{\dp0}
      } \end{minipage} \\
      \\ \cdashline{2-3}

      & \begin{minipage}[t]{2cm}{Actual\\ Result}\end{minipage}   & 
      \begin{minipage}[t]{13cm}{\footnotesize
      
      \vspace{\dp0}
      } \end{minipage} \\
      \\ \cdashline{2-3}


      & Status          & Not Executed \\ \hline

      6 & Description &

      \begin{minipage}[t]{13cm}{\footnotesize
      Follow Section 3.4.5.2 of the LSST Hexapods-Rotator Acceptance Test
Procedure, Sheet 49.

      \vspace{\dp0}
      } \end{minipage} \\
      \\ \cdashline{2-3}


      & Expected Result &

      \begin{minipage}[t]{13cm}{\footnotesize
      The Camera Rotator is able to reach the required acceleration of 1.0
degrees/sec\^{}2 as verified before per the LSST Hexapods\_Rotator
Acceptance Test Report, Sheet 52.

      \vspace{\dp0}
      } \end{minipage} \\
      \\ \cdashline{2-3}

      & \begin{minipage}[t]{2cm}{Actual\\ Result}\end{minipage}   & 
      \begin{minipage}[t]{13cm}{\footnotesize
      
      \vspace{\dp0}
      } \end{minipage} \\
      \\ \cdashline{2-3}


      & Status          & Not Executed \\ \hline

      7 & Description &

      \begin{minipage}[t]{13cm}{\footnotesize
      Follow Section 3.4.5.3 of the LSST Hexapods-Rotator Acceptance Test
Procedure, Sheet 49.

      \vspace{\dp0}
      } \end{minipage} \\
      \\ \cdashline{2-3}


      & Expected Result &

      \begin{minipage}[t]{13cm}{\footnotesize
      The initial result of the test (as seen in LSST Hexapods\_Rotator
Acceptance Test Report, Sheet 52-54) found that the requirement was not
met, but was accepted
\url{https://jira.lsstcorp.org/browse/LVV-7218}\emph{~}

      \vspace{\dp0}
      } \end{minipage} \\
      \\ \cdashline{2-3}

      & \begin{minipage}[t]{2cm}{Actual\\ Result}\end{minipage}   & 
      \begin{minipage}[t]{13cm}{\footnotesize
      
      \vspace{\dp0}
      } \end{minipage} \\
      \\ \cdashline{2-3}


      & Status          & Not Executed \\ \hline

      8 & Description &

      \begin{minipage}[t]{13cm}{\footnotesize
      Follow Section 3.4.5.4 of the LSST Hexapods-Rotator Acceptance Test
Procedure, Sheet 49.

      \vspace{\dp0}
      } \end{minipage} \\
      \\ \cdashline{2-3}


      & Expected Result &

      \begin{minipage}[t]{13cm}{\footnotesize
      
      \vspace{\dp0}
      } \end{minipage} \\
      \\ \cdashline{2-3}

      & \begin{minipage}[t]{2cm}{Actual\\ Result}\end{minipage}   & 
      \begin{minipage}[t]{13cm}{\footnotesize
      
      \vspace{\dp0}
      } \end{minipage} \\
      \\ \cdashline{2-3}


      & Status          & Not Executed \\ \hline

      9 & Description &

      \begin{minipage}[t]{13cm}{\footnotesize
      Follow Section 3.4.5.5 of the LSST Hexapods-Rotator Acceptance Test
Procedure, Sheet 50.

      \vspace{\dp0}
      } \end{minipage} \\
      \\ \cdashline{2-3}


      & Expected Result &

      \begin{minipage}[t]{13cm}{\footnotesize
      
      \vspace{\dp0}
      } \end{minipage} \\
      \\ \cdashline{2-3}

      & \begin{minipage}[t]{2cm}{Actual\\ Result}\end{minipage}   & 
      \begin{minipage}[t]{13cm}{\footnotesize
      
      \vspace{\dp0}
      } \end{minipage} \\
      \\ \cdashline{2-3}


      & Status          & Not Executed \\ \hline

      10 & Description &

      \begin{minipage}[t]{13cm}{\footnotesize
      Follow Section 3.4.5.6 of the LSST Hexapods-Rotator Acceptance Test
Procedure, Sheet 50.

      \vspace{\dp0}
      } \end{minipage} \\
      \\ \cdashline{2-3}


      & Expected Result &

      \begin{minipage}[t]{13cm}{\footnotesize
      
      \vspace{\dp0}
      } \end{minipage} \\
      \\ \cdashline{2-3}

      & \begin{minipage}[t]{2cm}{Actual\\ Result}\end{minipage}   & 
      \begin{minipage}[t]{13cm}{\footnotesize
      
      \vspace{\dp0}
      } \end{minipage} \\
      \\ \cdashline{2-3}


      & Status          & Not Executed \\ \hline

      11 & Description &

      \begin{minipage}[t]{13cm}{\footnotesize
      To verify 3.4.6.1, follow Section 3.4.6.2 of the LSST Hexapods-Rotator
Acceptance Test Procedure, Sheet 50-52.

      \vspace{\dp0}
      } \end{minipage} \\
      \\ \cdashline{2-3}


      & Expected Result &

      \begin{minipage}[t]{13cm}{\footnotesize
      The SN02 Rotator is found to be able to reach all of the testing speeds
from 0.005 to 0.068deg/s with a zero degree elevation angle.~

      \vspace{\dp0}
      } \end{minipage} \\
      \\ \cdashline{2-3}

      & \begin{minipage}[t]{2cm}{Actual\\ Result}\end{minipage}   & 
      \begin{minipage}[t]{13cm}{\footnotesize
      
      \vspace{\dp0}
      } \end{minipage} \\
      \\ \cdashline{2-3}


      & Status          & Not Executed \\ \hline

      12 & Description &

      \begin{minipage}[t]{13cm}{\footnotesize
      Section 3.4.6.2 will be tested with 3.4.6.1.

      \vspace{\dp0}
      } \end{minipage} \\
      \\ \cdashline{2-3}


      & Expected Result &

      \begin{minipage}[t]{13cm}{\footnotesize
      The SN02 Rotator's Tracking Accuracy is recorded for all tracking
velocities from 0.05deg/s to 0.068deg/s with a zero degree elevation
angle and is found to have a position error equal or better than 0.1
arcs seconds RMS.

      \vspace{\dp0}
      } \end{minipage} \\
      \\ \cdashline{2-3}

      & \begin{minipage}[t]{2cm}{Actual\\ Result}\end{minipage}   & 
      \begin{minipage}[t]{13cm}{\footnotesize
      
      \vspace{\dp0}
      } \end{minipage} \\
      \\ \cdashline{2-3}


      & Status          & Not Executed \\ \hline

      13 & Description &

      \begin{minipage}[t]{13cm}{\footnotesize
      See Section 3.4.6.6 of the LSST Hexapods-Rotator Acceptance Test
Procedure, Sheet 53

      \vspace{\dp0}
      } \end{minipage} \\
      \\ \cdashline{2-3}


      & Expected Result &

      \begin{minipage}[t]{13cm}{\footnotesize
      By checking the current of the system, the heat dissipation for the
rotator is verified to be less than 40W.

      \vspace{\dp0}
      } \end{minipage} \\
      \\ \cdashline{2-3}

      & \begin{minipage}[t]{2cm}{Actual\\ Result}\end{minipage}   & 
      \begin{minipage}[t]{13cm}{\footnotesize
      
      \vspace{\dp0}
      } \end{minipage} \\
      \\ \cdashline{2-3}


      & Status          & Not Executed \\ \hline

      14 & Description &

      \begin{minipage}[t]{13cm}{\footnotesize
      Follow Section 3.4.9 of the LSST Hexapods-Rotator Acceptance Test
Procedure, Sheet 54.

      \vspace{\dp0}
      } \end{minipage} \\
      \\ \cdashline{2-3}


      & Expected Result &

      \begin{minipage}[t]{13cm}{\footnotesize
      The locking pin is demonstrated to be able to engage at 15 deg intervals
throughout the entire rotator range while the camera hexapod/rotator is
installed on the camera cart.

      \vspace{\dp0}
      } \end{minipage} \\
      \\ \cdashline{2-3}

      & \begin{minipage}[t]{2cm}{Actual\\ Result}\end{minipage}   & 
      \begin{minipage}[t]{13cm}{\footnotesize
      
      \vspace{\dp0}
      } \end{minipage} \\
      \\ \cdashline{2-3}


      & Status          & Not Executed \\ \hline

    \end{longtable}


\newpage
\appendix
%Make sure lsst-texmf/bin/generateAcronyms.py is in your path
\section{Acronyms used in this document}\label{sec:acronyms}
\addtocounter{table}{-1}
\begin{longtable}{p{0.145\textwidth}p{0.8\textwidth}}\hline
\textbf{Acronym} & \textbf{Description}  \\\hline

EFD & Engineering and Facility Database \\\hline
GUI & Graphical User Interface \\\hline
LSST & Large Synoptic Survey Telescope \\\hline
PMCS & Project Management Controls System \\\hline
RMS & Root-Mean-Square \\\hline
SLAC & SLAC National Accelerator Laboratory (formerly Stanford Linear Accelerator Center; SLAC is now no longer an acronym) \\\hline
\end{longtable}


\end{document}
