\documentclass[SE,lsstdraft,STR,toc]{lsstdoc}
\usepackage{geometry}
\usepackage{longtable,booktabs}
\usepackage{enumitem}
\usepackage{arydshln}

\input meta.tex

\providecommand{\tightlist}{
  \setlength{\itemsep}{0pt}\setlength{\parskip}{0pt}}

\begin{document}

\def\milestoneName{Camera Rotator Functional Re-Verification}
\def\milestoneId{LVV-P59}
\def\product{SIT-COM Integration}

\setDocCompact{true}

\title{ LVV-P59 Camera Rotator Functional Re-Verification Test Plan and Report}
\setDocRef{\lsstDocType-\lsstDocNum}
\date{\vcsdate}
\setDocUpstreamLocation{\url{https://github.com/lsst/lsst-texmf/examples}}
\author{ Kevin Siruno }

\input history_and_info.tex


\setDocAbstract{
This is the test plan and report for LVV-P59 (Camera Rotator Functional Re-Verification),
an LSST milestone pertaining to the System Engineering Subsystem.
}


\maketitle

\section{Introduction}
\label{sect:intro}


\subsection{Objectives}
\label{sect:objectives}

The objective of this test plan is to re-verify the functional
requirements of the camera rotator's hardware and software, after
shipment from the vendors facility to the Summit, as defined in \citeds{LTS-206}
and \citeds{LTS-160}. This test campaign will only exercise the functionality
that was executed previously and meets the following criteria:

\begin{itemize}
\tightlist
\item
  Only requires the camera rotator to be operable
\item
  Only requires the vendors EUI software and hardware via local control
\item
  Only requires a laser tracker
\item
  Does \textbf{NOT} require the camera rotator to be loaded with the
  camera simulated mass or actual camera hardware
\end{itemize}

The hardware functional requirements were previously verified during the
test campaign by the vendor at the vendors facility and accepted by LSST
during the Factory Acceptance Test review.



\subsection{System Overview}
\label{sect:systemoverview}

The Camera Rotator is mounted to the Camera Hexapod with the primary
function of rotating the camera about the Camera's central axis.


\subsection{Document Overview}
\label{sect:docoverview}

This document was generated from Jira, obtaining the relevant information from the 
\href{https://jira.lsstcorp.org/secure/Tests.jspa#/testPlan/LVV-P59}{LVV-P59}
~Jira Test Plan and related Test Cycles (
  \href{https://jira.lsstcorp.org/secure/Tests.jspa#/testCycle/LVV-C110}{LVV-C110}
).

Section \ref{sect:intro} provides an overview of the test campaign, the system under test (\product{}), the applicable documentation, and explains how this document is organized.
Section \ref{sect:configuration}  describes the configuration used for this test.
Section \ref{sect:personnel} describes the necessary roles and lists the individuals assigned to them.
%Section \ref{sect:plannedtestactivities} provides the list of planned test cycles and test cases, including all relevant information that fully describes the test campaign.

Section \ref{sect:overview} provides a summary of the test results, including an overview in Table \ref{table:summary}, an overall assessment statement and suggestions for possible improvements.
Section \ref{sect:detailedtestresults} provides detailed results for each step in each test case.

The current status of test plan LVV-P59 in Jira is \textbf{ Draft }.

\subsection{References}
\label{sect:references}
\renewcommand{\refname}{}
\bibliography{lsst,refs,books,refs_ads,local}
\section{Test Configuration}
\label{sect:configuration}

\subsection{Data Collection}

  Observing is not required for this test campaign.

\subsection{Verification Environment}
\label{sect:hwconf}
  The Camera Rotator will be verified in a climate controlled environment
on the 3rd floor of the Summit Facility integrated with the Camera Cable
Wrap on the Camera Cart.


  \subsection{Entry Criteria}
  In order to test the Camera Rotator functionality, the following
criteria must be met first:

\begin{itemize}
\tightlist
\item
  All the test setup for the Data Acquisition system must be completed
  and ready to record data for the laser tracker and current probes
\item
  The Laser tracker and SMR's are installed and setup
\item
  The Inductive current probes are installed and setup
\item
  All utilities and electrical connections are hooked up and allow the
  Camera Rotator to be powered on and controlled
\item
  The EFD must be set up to be able to store events and telemetry data
\end{itemize}


  \subsection{Exit Criteria}
  In order for this event to be considered complete, the following
criteria must be met:

\begin{itemize}
\tightlist
\item
  Raw test data, events, and telemetry have been saved for the Camera
  Rotator.
\item
  All test data has been analyzed and post processed.
\item
  All test steps have been statused in the Jira Test Cases within this
  Test Plan and actual results populated as required.
\item
  A summary of the results of the test campaign has been captured in the
  Overall Assessment and Recommended Improvements fields of this Test
  Plan
\item
  A link to the verification artifacts used to produce the summary of
  results has been populated in the Verification Artifacts field of this
  Test Plan
\item
  Any failures have been captured in the
  \href{https://jira.lsstcorp.org/projects/FRACAS/issues/}{FRACAS}
  project
\end{itemize}


  \subsection{PMCS Activity}
  See Epics in Traceability Tab


\newpage
\section{Personnel}
\label{sect:personnel}

The personnel involved in the test campaign is shown in the following table.

\begin{longtable}{p{3cm}p{3cm}p{3cm}p{6cm}}
\hline
\multicolumn{2}{r}{Test Plan (LVV-P59) owner:} &
\multicolumn{2}{l}{\textbf{ Kevin Siruno } }\\\hline
\multicolumn{2}{r}{ LVV-C110 owner:} &
\multicolumn{2}{l}{\textbf{
    Kevin Siruno
}
} \\\hline
\textbf{Test Case} & \textbf{Assigned to} & \textbf{Executed by} & \textbf{Additional Test Personnel} \\ \hline
\href{https://jira.lsstcorp.org/secure/Tests.jspa#/testCase/LVV-T1577}{LVV-T1577}
& {\small Kevin Siruno } & {\small  } &
\begin{minipage}[]{6cm}
\smallskip
{\small (1) Software Engineer\\
(1) Hardware Engineer
 }
\medskip
\end{minipage}
\\ \hline
\href{https://jira.lsstcorp.org/secure/Tests.jspa#/testCase/LVV-T1576}{LVV-T1576}
& {\small Kevin Siruno } & {\small  } &
\begin{minipage}[]{6cm}
\smallskip
{\small (1) Mechanical Engineer/Optical Engineer\\
(1) Electrical Engineer
 }
\medskip
\end{minipage}
\\ \hline
\end{longtable}

\newpage

\section{Test Campaign Overview}
\label{sect:overview}

\subsection{Summary}
\label{sect:summarytable}

\begin{longtable}{p{2.5cm}p{3cm}p{7.5cm}p{3cm}}
\toprule
\multicolumn{3}{l}{ Test Plan {\bf LVV-P59: Camera Rotator Functional Re-verification
 }} & Draft \\\hline

  \multicolumn{3}{l}{ Test Cycle {\bf LVV-C110: Camera Rotator Re-verification
 }} & Not Executed \\\hline

  {\bf \footnotesize test case} & {\bf \footnotesize status} & {\bf \footnotesize comment} & {\bf \footnotesize issues} \\\toprule

    \href{https://jira.lsstcorp.org/secure/Tests.jspa#/testCase/LVV-T1577}{LVV-T1577}
    & Not Executed &
    \begin{minipage}[]{9cm}
    \smallskip
    
    \medskip
    \end{minipage}
    &
    \\\hline
    \href{https://jira.lsstcorp.org/secure/Tests.jspa#/testCase/LVV-T1576}{LVV-T1576}
    & Not Executed &
    \begin{minipage}[]{9cm}
    \smallskip
    
    \medskip
    \end{minipage}
    &
    \\\hline
\caption{Test Results Summary}
\label{table:summary}
\end{longtable}

\subsection{Overall Assessment}
\label{sect:overallassessment}

Not yet available.

\subsection{Recommended Improvements}
\label{sect:recommendations}

Not yet available.

\newpage
\section{Detailed Test Results}
\label{sect:detailedtestresults}

\subsection{Test Cycle LVV-C110 }

Open test cycle {\it \href{https://jira.lsstcorp.org/secure/Tests.jspa#/testrun/LVV-C110}{Camera Rotator Re-verification
}} in Jira.

Camera Rotator Re-verification
\\
Status: Not Executed

Re-verify the hardware and software requirements for the Camera rotator
that were previously tested by MOOG.


\subsubsection{Software Version/Baseline}
\begin{enumerate}
\tightlist
\item
  Camera Rotator Control Software with SAL v3.5
\item
  Mariadb EFD with SAL v3.5
\end{enumerate}


\subsubsection{Configuration}
No varying configuration between test cycles.


\subsubsection{Test Cases in LVV-C110 Test Cycle}

\paragraph{Test Case LVV-T1577 - Camera Rotator Software Functional Re-verification
 }\mbox{}\\

Open  \href{https://jira.lsstcorp.org/secure/Tests.jspa#/testCase/LVV-T1577}{\textit{ LVV-T1577 } }
test case in Jira.

The objective of this test case is to re-verify the functional
requirements of the camera rotator's software, after shipment of the
hardware from the vendor's facility to the Summit, as defined in \citeds{LTS-206}
and \citeds{LTS-160}. This test case will only exercise the functionality that
was executed previously and meets the following criteria:

\begin{itemize}
\tightlist
\item
  Only requires the camera rotator to be operable
\item
  Only requires the vendors EUI software and hardware via local control
\item
  Does \textbf{NOT} require the camera rotator to be loaded with the
  camera simulated mass or actual camera hardware
\end{itemize}

The software functional requirements were previously verified during the
test campaign by the vendor at the vendor's facility and accepted by
LSST during the Factory Acceptance Test review. The test procedure used
during the vendor's acceptance testing is the \emph{LSST
Hexapods-Rotator Software Acceptance Test Procedure} which is attached
to this test case. The test steps of this test case are taken directly
from that document on how to perform the test in a similar way as was
performed previously and includes changes noted by the
vendor.\\[2\baselineskip]The \emph{LSST Hexapods-Rotator Software
Acceptance Test Procedure\_Completed} has also been attached to this
test case for additional information on which tests were
passed.\\[2\baselineskip]See the Confluence page \emph{Updated
Middleware Test in SLAC at 10/15/2019} linked in the Traceability Tab
for additional results of the State Transition Test done by LSST using a
GUI and SAL 3.10.\\[2\baselineskip]See the attached \emph{LSST Rotator
Operator's Manual} for more information on how to operate the
rotator.\\[2\baselineskip]


\textbf{ Preconditions}:\\
Prior to the execution of this test case to re-verify the Camera Rotator
software functional requirements, the following Summit tasks must be
completed:\\

\begin{itemize}
\tightlist
\item
  The cables and cabinets have been checked~

  \begin{itemize}
  \tightlist
  \item
    \url{https://jira.lsstcorp.org/browse/SUMMIT-3231}
  \end{itemize}
\item
  Boxes for the hexapod/rotator have been transported to the 3rd level

  \begin{itemize}
  \tightlist
  \item
    \url{https://jira.lsstcorp.org/browse/SUMMIT-3230}
  \end{itemize}
\item
  Test fit Camera Hexapod with Offset

  \begin{itemize}
  \tightlist
  \item
    \url{https://jira.lsstcorp.org/browse/SUMMIT-3293}
  \end{itemize}
\item
  The Hexapod and Rotator have been installed on camera cart

  \begin{itemize}
  \tightlist
  \item
    \url{https://jira.lsstcorp.org/browse/SUMMIT-3224}
  \end{itemize}
\item
  The Camera hexapod/rotator has been connected to the electronics
  cabinets and the connections have been tested

  \begin{itemize}
  \tightlist
  \item
    \url{https://jira.lsstcorp.org/browse/SUMMIT-3294}
  \end{itemize}
\end{itemize}


Execution status: {\bf Not Executed }

Final comment:\\


Detailed steps results:

\begin{longtable}{p{1cm}p{15cm}}
\hline
{Step} & Step Details\\ \hline
1 & Description \\
 & \begin{minipage}[t]{15cm}
{\footnotesize
\smallskip
\textbf{STARTING THE EUI}\\[2\baselineskip]Double click the Hexapod GUI
Viewer desktop icon on the computer.

\begin{itemize}
\tightlist
\item
  This can be done on the Dell Management PC or another computer on the
  same network
\end{itemize}

\medskip }
\end{minipage}
\\ \cdashline{2-2}


 & Expected Result \\
 & \begin{minipage}[t]{15cm}{\footnotesize
\smallskip
A prompt to enter the password is shown.

\medskip }
\end{minipage} \\ \cdashline{2-2}

 & Actual Result \\
 & \begin{minipage}[t]{15cm}{\footnotesize
\smallskip

\medskip }
\end{minipage} \\ \cdashline{2-2}

 & Status: \textbf{ Not Executed } \\ \hline

2 & Description \\
 & \begin{minipage}[t]{15cm}
{\footnotesize
\smallskip
Enter the password ``lsst-vnc''

\begin{itemize}
\tightlist
\item
  If the EUI isn't automatically up and running when the VNC opens,
  double click on the CAM\_Hex\_eGUI or M2\_Hex\_eGUI icon on the VNC
  viewer
\end{itemize}

\medskip }
\end{minipage}
\\ \cdashline{2-2}


 & Expected Result \\
 & \begin{minipage}[t]{15cm}{\footnotesize
\smallskip
The EUI is in the Offline State/PublishOnly substate and is able to
publish through SAL but cannot receive commands.

\medskip }
\end{minipage} \\ \cdashline{2-2}

 & Actual Result \\
 & \begin{minipage}[t]{15cm}{\footnotesize
\smallskip

\medskip }
\end{minipage} \\ \cdashline{2-2}

 & Status: \textbf{ Not Executed } \\ \hline

3 & Description \\
 & \begin{minipage}[t]{15cm}
{\footnotesize
\smallskip
\textbf{OFFLINESTATE/AVAILABLESTATE}\\
On the Main tab, select the ``Offline SubState Cmd'' field in the
Commands to Send section, set the Offline SubState Triggers to ``System
Ready'' and click on the Send Command button.\\
\includegraphics[width=1.79167in]{jira_imgs/1005.png}

\medskip }
\end{minipage}
\\ \cdashline{2-2}


 & Expected Result \\
 & \begin{minipage}[t]{15cm}{\footnotesize
\smallskip
The system transitions from the OfflineState/PublishOnly substate to the
OfflineState/AvailableState
substate.\\[2\baselineskip]\includegraphics[width=3.12500in]{jira_imgs/1007.png}

\medskip }
\end{minipage} \\ \cdashline{2-2}

 & Actual Result \\
 & \begin{minipage}[t]{15cm}{\footnotesize
\smallskip

\medskip }
\end{minipage} \\ \cdashline{2-2}

 & Status: \textbf{ Not Executed } \\ \hline

4 & Description \\
 & \begin{minipage}[t]{15cm}
{\footnotesize
\smallskip
\textbf{OFFLINESTATE -\textgreater{} STANDBYSTATE}\\
Click on the State Command field in the Commands to Send section.\\
\includegraphics[width=1.79167in]{jira_imgs/1030.png}

\medskip }
\end{minipage}
\\ \cdashline{2-2}


 & Expected Result \\
 & \begin{minipage}[t]{15cm}{\footnotesize
\smallskip
The system transitions into the StandbyState and the primary state
display box at the top of the Main tab says Standby State.\\
\includegraphics[width=4.68750in]{jira_imgs/1018.png}

\medskip }
\end{minipage} \\ \cdashline{2-2}

 & Actual Result \\
 & \begin{minipage}[t]{15cm}{\footnotesize
\smallskip

\medskip }
\end{minipage} \\ \cdashline{2-2}

 & Status: \textbf{ Not Executed } \\ \hline

5 & Description \\
 & \begin{minipage}[t]{15cm}
{\footnotesize
\smallskip
\textbf{STANDBYSTATE -\textgreater{} DISABLEDSTATE}\\
From the StandbyState, send a start command.

\medskip }
\end{minipage}
\\ \cdashline{2-2}


 & Expected Result \\
 & \begin{minipage}[t]{15cm}{\footnotesize
\smallskip
The system transitions into DisabledState and the current configuration
parameters are maintained from the default parameters or from the
previous DDS start command.~\\
\includegraphics[width=3.12500in]{jira_imgs/1019.png}\\
If the configuration file is invalid or out of range, the system will
transition into a Fault State

\medskip }
\end{minipage} \\ \cdashline{2-2}

 & Actual Result \\
 & \begin{minipage}[t]{15cm}{\footnotesize
\smallskip

\medskip }
\end{minipage} \\ \cdashline{2-2}

 & Status: \textbf{ Not Executed } \\ \hline

6 & Description \\
 & \begin{minipage}[t]{15cm}
{\footnotesize
\smallskip
\textbf{DISABLEDSTATE -\textgreater{} ENABLEDSTATE}\\
From the DisabledState, send an Enable state.

\medskip }
\end{minipage}
\\ \cdashline{2-2}


 & Expected Result \\
 & \begin{minipage}[t]{15cm}{\footnotesize
\smallskip
The system transitions into the EnabledState/Stationary substate, the
motor drives are enabled, and motion can be commanded.\\
\includegraphics[width=3.12500in]{jira_imgs/1020.png}\\

\medskip }
\end{minipage} \\ \cdashline{2-2}

 & Actual Result \\
 & \begin{minipage}[t]{15cm}{\footnotesize
\smallskip

\medskip }
\end{minipage} \\ \cdashline{2-2}

 & Status: \textbf{ Not Executed } \\ \hline

7 & Description \\
 & \begin{minipage}[t]{15cm}
{\footnotesize
\smallskip
\textbf{FAULTSTATE}\\
If a Fault occurs in any of the other states, the system will
automatically transition to the Fault State. While in the Fault state,
send a clearError command.\\
Note: If the fault that occurs goes through the interlock system, reset
the safety relay switch and send a clearError command.

\medskip }
\end{minipage}
\\ \cdashline{2-2}


 & Expected Result \\
 & \begin{minipage}[t]{15cm}{\footnotesize
\smallskip
The system transitions back to the OfflineState/PublishOnly substate.
(Go back to Step 3)\\
\includegraphics[width=3.12500in]{jira_imgs/1021.png}

\medskip }
\end{minipage} \\ \cdashline{2-2}

 & Actual Result \\
 & \begin{minipage}[t]{15cm}{\footnotesize
\smallskip

\medskip }
\end{minipage} \\ \cdashline{2-2}

 & Status: \textbf{ Not Executed } \\ \hline

8 & Description \\
 & \begin{minipage}[t]{15cm}
{\footnotesize
\smallskip
\textbf{Section 3.2.1 of the attached Software Acceptance Test
Procedure\\
Test Sequence \#1 - PositionSet and Move Commands}\\[2\baselineskip]In
the Enabled/Stationary state, send a positionSet command of 12 deg.

\medskip }
\end{minipage}
\\ \cdashline{2-2}


 & Expected Result \\
 & \begin{minipage}[t]{15cm}{\footnotesize
\smallskip
Confirm that the rotator does not move.

\medskip }
\end{minipage} \\ \cdashline{2-2}

 & Actual Result \\
 & \begin{minipage}[t]{15cm}{\footnotesize
\smallskip

\medskip }
\end{minipage} \\ \cdashline{2-2}

 & Status: \textbf{ Not Executed } \\ \hline

9 & Description \\
 & \begin{minipage}[t]{15cm}
{\footnotesize
\smallskip
Send a positionSet command of 15deg.

\medskip }
\end{minipage}
\\ \cdashline{2-2}


 & Expected Result \\
 & \begin{minipage}[t]{15cm}{\footnotesize
\smallskip
Confirm that the rotator does not move.

\medskip }
\end{minipage} \\ \cdashline{2-2}

 & Actual Result \\
 & \begin{minipage}[t]{15cm}{\footnotesize
\smallskip

\medskip }
\end{minipage} \\ \cdashline{2-2}

 & Status: \textbf{ Not Executed } \\ \hline

10 & Description \\
 & \begin{minipage}[t]{15cm}
{\footnotesize
\smallskip
Send a move command.

\medskip }
\end{minipage}
\\ \cdashline{2-2}


 & Expected Result \\
 & \begin{minipage}[t]{15cm}{\footnotesize
\smallskip
Confirm that the rotator moves to 15 deg.

\medskip }
\end{minipage} \\ \cdashline{2-2}

 & Actual Result \\
 & \begin{minipage}[t]{15cm}{\footnotesize
\smallskip

\medskip }
\end{minipage} \\ \cdashline{2-2}

 & Status: \textbf{ Not Executed } \\ \hline

11 & Description \\
 & \begin{minipage}[t]{15cm}
{\footnotesize
\smallskip
\textbf{Section 3.2.1 of the attached Software Acceptance Test
Procedure\\
Test Sequence \#2 - StopCommand}\\[2\baselineskip]In the
Enabled/Stationary state, send a positionSet command of 50 deg.

\medskip }
\end{minipage}
\\ \cdashline{2-2}


 & Expected Result \\
 & \begin{minipage}[t]{15cm}{\footnotesize
\smallskip
Confirm that the rotator does not move.

\medskip }
\end{minipage} \\ \cdashline{2-2}

 & Actual Result \\
 & \begin{minipage}[t]{15cm}{\footnotesize
\smallskip

\medskip }
\end{minipage} \\ \cdashline{2-2}

 & Status: \textbf{ Not Executed } \\ \hline

12 & Description \\
 & \begin{minipage}[t]{15cm}
{\footnotesize
\smallskip
Send a move command.

\medskip }
\end{minipage}
\\ \cdashline{2-2}


 & Expected Result \\
 & \begin{minipage}[t]{15cm}{\footnotesize
\smallskip
The rotator starts it's rotation.

\medskip }
\end{minipage} \\ \cdashline{2-2}

 & Actual Result \\
 & \begin{minipage}[t]{15cm}{\footnotesize
\smallskip

\medskip }
\end{minipage} \\ \cdashline{2-2}

 & Status: \textbf{ Not Executed } \\ \hline

13 & Description \\
 & \begin{minipage}[t]{15cm}
{\footnotesize
\smallskip
While the rotator is still moving, send a Stop command.

\medskip }
\end{minipage}
\\ \cdashline{2-2}


 & Expected Result \\
 & \begin{minipage}[t]{15cm}{\footnotesize
\smallskip
Confirm that the system quickly comes to a stop before reaching the 50
deg position.

\medskip }
\end{minipage} \\ \cdashline{2-2}

 & Actual Result \\
 & \begin{minipage}[t]{15cm}{\footnotesize
\smallskip

\medskip }
\end{minipage} \\ \cdashline{2-2}

 & Status: \textbf{ Not Executed } \\ \hline

14 & Description \\
 & \begin{minipage}[t]{15cm}
{\footnotesize
\smallskip
Send a positionSet command of 60 deg followed by a move command.

\medskip }
\end{minipage}
\\ \cdashline{2-2}


 & Expected Result \\
 & \begin{minipage}[t]{15cm}{\footnotesize
\smallskip
Confirm the rotator moves to the commanded position following the
previous stop command.

\medskip }
\end{minipage} \\ \cdashline{2-2}

 & Actual Result \\
 & \begin{minipage}[t]{15cm}{\footnotesize
\smallskip

\medskip }
\end{minipage} \\ \cdashline{2-2}

 & Status: \textbf{ Not Executed } \\ \hline

15 & Description \\
 & \begin{minipage}[t]{15cm}
{\footnotesize
\smallskip
\textbf{Section 3.2.1 of the attached Software Acceptance Test
Procedure\\
Test Sequence \#3 - VelocitySet and MoveConstantVelocity
Commands}\\[2\baselineskip]In the Enabled/Stationary state, send a
velocitySet command of 0.01 deg/s and 15 seconds.

\medskip }
\end{minipage}
\\ \cdashline{2-2}


 & Expected Result \\
 & \begin{minipage}[t]{15cm}{\footnotesize
\smallskip
Confirm that the rotator does not move.

\medskip }
\end{minipage} \\ \cdashline{2-2}

 & Actual Result \\
 & \begin{minipage}[t]{15cm}{\footnotesize
\smallskip

\medskip }
\end{minipage} \\ \cdashline{2-2}

 & Status: \textbf{ Not Executed } \\ \hline

16 & Description \\
 & \begin{minipage}[t]{15cm}
{\footnotesize
\smallskip
Send a velocitySet command of 0.02 deg/s and 10 seconds.

\medskip }
\end{minipage}
\\ \cdashline{2-2}


 & Expected Result \\
 & \begin{minipage}[t]{15cm}{\footnotesize
\smallskip
Confirm that the rotator does not move.

\medskip }
\end{minipage} \\ \cdashline{2-2}

 & Actual Result \\
 & \begin{minipage}[t]{15cm}{\footnotesize
\smallskip

\medskip }
\end{minipage} \\ \cdashline{2-2}

 & Status: \textbf{ Not Executed } \\ \hline

17 & Description \\
 & \begin{minipage}[t]{15cm}
{\footnotesize
\smallskip
Send a moveConstantVelocity command.

\medskip }
\end{minipage}
\\ \cdashline{2-2}


 & Expected Result \\
 & \begin{minipage}[t]{15cm}{\footnotesize
\smallskip
Confirm that the rotator moves approximately 0.2 degrees total over 10
seconds before coming to a stop and transitioning back to
enabled/stationary state.

\medskip }
\end{minipage} \\ \cdashline{2-2}

 & Actual Result \\
 & \begin{minipage}[t]{15cm}{\footnotesize
\smallskip

\medskip }
\end{minipage} \\ \cdashline{2-2}

 & Status: \textbf{ Not Executed } \\ \hline

18 & Description \\
 & \begin{minipage}[t]{15cm}
{\footnotesize
\smallskip
\textbf{Test of the Velocity Limit}\\[2\baselineskip]In the
enabled/stationary state, send a velocitySet command of 4deg/s and 5
seconds through the EUI.

\medskip }
\end{minipage}
\\ \cdashline{2-2}


 & Expected Result \\
 & \begin{minipage}[t]{15cm}{\footnotesize
\smallskip
The EUI does not allow for any value higher than 3.5 deg/s as an input
for the velocitySet command.\\
\emph{Note: If the EUI does not reject the velocitySet command, go to
Step 12.}

\medskip }
\end{minipage} \\ \cdashline{2-2}

 & Actual Result \\
 & \begin{minipage}[t]{15cm}{\footnotesize
\smallskip

\medskip }
\end{minipage} \\ \cdashline{2-2}

 & Status: \textbf{ Not Executed } \\ \hline

19 & Description \\
 & \begin{minipage}[t]{15cm}
{\footnotesize
\smallskip
\textbf{\emph{\textless{}Conditional
Step\textgreater{}}}\\[2\baselineskip]If the EUI accepts the value of
4deg/s and 5 seconds, send a move command.

\medskip }
\end{minipage}
\\ \cdashline{2-2}


 & Expected Result \\
 & \begin{minipage}[t]{15cm}{\footnotesize
\smallskip
If the rotator does not ignore the velocitySet command, the rotator does
not move at a higher velocity than the velocity limit of 3.5 deg/s.

\medskip }
\end{minipage} \\ \cdashline{2-2}

 & Actual Result \\
 & \begin{minipage}[t]{15cm}{\footnotesize
\smallskip

\medskip }
\end{minipage} \\ \cdashline{2-2}

 & Status: \textbf{ Not Executed } \\ \hline

20 & Description \\
 & \begin{minipage}[t]{15cm}
{\footnotesize
\smallskip
Perform a shutdown of the rotator.

\medskip }
\end{minipage}
\\ \cdashline{2-2}


 & Expected Result \\
 & \begin{minipage}[t]{15cm}{\footnotesize
\smallskip
Rotator is shutdown.

\medskip }
\end{minipage} \\ \cdashline{2-2}

 & Actual Result \\
 & \begin{minipage}[t]{15cm}{\footnotesize
\smallskip

\medskip }
\end{minipage} \\ \cdashline{2-2}

 & Status: \textbf{ Not Executed } \\ \hline

21 & Description \\
 & \begin{minipage}[t]{15cm}
{\footnotesize
\smallskip
\textbf{Section 3.3.1 of the attached Software Acceptance Test
Procedure}\\
\textbf{Actions on State Commands\\
}\\
Startup the rotator.

\medskip }
\end{minipage}
\\ \cdashline{2-2}


 & Expected Result \\
 & \begin{minipage}[t]{15cm}{\footnotesize
\smallskip
Confirm that the rotator starts in the Offline/PublishOnly state.

\medskip }
\end{minipage} \\ \cdashline{2-2}

 & Actual Result \\
 & \begin{minipage}[t]{15cm}{\footnotesize
\smallskip

\medskip }
\end{minipage} \\ \cdashline{2-2}

 & Status: \textbf{ Not Executed } \\ \hline

22 & Description \\
 & \begin{minipage}[t]{15cm}
{\footnotesize
\smallskip
Send an Offline substate trigger of systemReady.

\medskip }
\end{minipage}
\\ \cdashline{2-2}


 & Expected Result \\
 & \begin{minipage}[t]{15cm}{\footnotesize
\smallskip
Confirm system goes into Offline/Available substate.

\medskip }
\end{minipage} \\ \cdashline{2-2}

 & Actual Result \\
 & \begin{minipage}[t]{15cm}{\footnotesize
\smallskip

\medskip }
\end{minipage} \\ \cdashline{2-2}

 & Status: \textbf{ Not Executed } \\ \hline

23 & Description \\
 & \begin{minipage}[t]{15cm}
{\footnotesize
\smallskip
Send an EnterControl trigger.

\medskip }
\end{minipage}
\\ \cdashline{2-2}


 & Expected Result \\
 & \begin{minipage}[t]{15cm}{\footnotesize
\smallskip
Confirm the system transitions from Offline/Available to Standby state.

\medskip }
\end{minipage} \\ \cdashline{2-2}

 & Actual Result \\
 & \begin{minipage}[t]{15cm}{\footnotesize
\smallskip

\medskip }
\end{minipage} \\ \cdashline{2-2}

 & Status: \textbf{ Not Executed } \\ \hline

24 & Description \\
 & \begin{minipage}[t]{15cm}
{\footnotesize
\smallskip
Send a Start trigger.

\medskip }
\end{minipage}
\\ \cdashline{2-2}


 & Expected Result \\
 & \begin{minipage}[t]{15cm}{\footnotesize
\smallskip
Confirm the system transitions from Standby to Disabled state.

\medskip }
\end{minipage} \\ \cdashline{2-2}

 & Actual Result \\
 & \begin{minipage}[t]{15cm}{\footnotesize
\smallskip

\medskip }
\end{minipage} \\ \cdashline{2-2}

 & Status: \textbf{ Not Executed } \\ \hline

25 & Description \\
 & \begin{minipage}[t]{15cm}
{\footnotesize
\smallskip
Send an Enable trigger.

\medskip }
\end{minipage}
\\ \cdashline{2-2}


 & Expected Result \\
 & \begin{minipage}[t]{15cm}{\footnotesize
\smallskip
Confirm the system transitions from Disabled to Enabled state.

\medskip }
\end{minipage} \\ \cdashline{2-2}

 & Actual Result \\
 & \begin{minipage}[t]{15cm}{\footnotesize
\smallskip

\medskip }
\end{minipage} \\ \cdashline{2-2}

 & Status: \textbf{ Not Executed } \\ \hline

26 & Description \\
 & \begin{minipage}[t]{15cm}
{\footnotesize
\smallskip
Send a Disable trigger.

\medskip }
\end{minipage}
\\ \cdashline{2-2}


 & Expected Result \\
 & \begin{minipage}[t]{15cm}{\footnotesize
\smallskip
Confirm the system transitions from Enabled to Disabled state.

\medskip }
\end{minipage} \\ \cdashline{2-2}

 & Actual Result \\
 & \begin{minipage}[t]{15cm}{\footnotesize
\smallskip

\medskip }
\end{minipage} \\ \cdashline{2-2}

 & Status: \textbf{ Not Executed } \\ \hline

27 & Description \\
 & \begin{minipage}[t]{15cm}
{\footnotesize
\smallskip
Send a Standby trigger.\\[2\baselineskip]

\medskip }
\end{minipage}
\\ \cdashline{2-2}


 & Expected Result \\
 & \begin{minipage}[t]{15cm}{\footnotesize
\smallskip
Confirm the system transitions from Disabled state to Standby state.

\medskip }
\end{minipage} \\ \cdashline{2-2}

 & Actual Result \\
 & \begin{minipage}[t]{15cm}{\footnotesize
\smallskip

\medskip }
\end{minipage} \\ \cdashline{2-2}

 & Status: \textbf{ Not Executed } \\ \hline

28 & Description \\
 & \begin{minipage}[t]{15cm}
{\footnotesize
\smallskip
Send a exitControl trigger.

\medskip }
\end{minipage}
\\ \cdashline{2-2}


 & Expected Result \\
 & \begin{minipage}[t]{15cm}{\footnotesize
\smallskip
Confirm the system transitions from Standby state to Offline state.

\medskip }
\end{minipage} \\ \cdashline{2-2}

 & Actual Result \\
 & \begin{minipage}[t]{15cm}{\footnotesize
\smallskip

\medskip }
\end{minipage} \\ \cdashline{2-2}

 & Status: \textbf{ Not Executed } \\ \hline

29 & Description \\
 & \begin{minipage}[t]{15cm}
{\footnotesize
\smallskip
Return to the Enabled state and trip the safety interlock switch.

\medskip }
\end{minipage}
\\ \cdashline{2-2}


 & Expected Result \\
 & \begin{minipage}[t]{15cm}{\footnotesize
\smallskip
Confirm the system transitions to Fault state.

\medskip }
\end{minipage} \\ \cdashline{2-2}

 & Actual Result \\
 & \begin{minipage}[t]{15cm}{\footnotesize
\smallskip

\medskip }
\end{minipage} \\ \cdashline{2-2}

 & Status: \textbf{ Not Executed } \\ \hline

30 & Description \\
 & \begin{minipage}[t]{15cm}
{\footnotesize
\smallskip
Reset the safety interlock and send a ClearError trigger.

\medskip }
\end{minipage}
\\ \cdashline{2-2}


 & Expected Result \\
 & \begin{minipage}[t]{15cm}{\footnotesize
\smallskip
Confirm the system transitions from Fault state to Offline state.

\medskip }
\end{minipage} \\ \cdashline{2-2}

 & Actual Result \\
 & \begin{minipage}[t]{15cm}{\footnotesize
\smallskip

\medskip }
\end{minipage} \\ \cdashline{2-2}

 & Status: \textbf{ Not Executed } \\ \hline

31 & Description \\
 & \begin{minipage}[t]{15cm}
{\footnotesize
\smallskip
\textbf{Section 5.1 of the attached Software Acceptance Test
Procedure}\\
\textbf{Rotator Events\\
}\\
In the Enabled state, unplug an encoder cable for one of the rotator
motors.

\medskip }
\end{minipage}
\\ \cdashline{2-2}

 & Test Data \\
 & \begin{minipage}[t]{15cm}{\footnotesize
\smallskip
\textbf{Deviation:~}Perform the following set of steps using the EUI
instead of the DDS and verify the events are displayed on the EUI.

\medskip }
\end{minipage} \\ \cdashline{2-2}

 & Expected Result \\
 & \begin{minipage}[t]{15cm}{\footnotesize
\smallskip
Confirm that a Drive Fault event is created and the system transitions
to Fault state.

\medskip }
\end{minipage} \\ \cdashline{2-2}

 & Actual Result \\
 & \begin{minipage}[t]{15cm}{\footnotesize
\smallskip

\medskip }
\end{minipage} \\ \cdashline{2-2}

 & Status: \textbf{ Not Executed } \\ \hline

32 & Description \\
 & \begin{minipage}[t]{15cm}
{\footnotesize
\smallskip
In the Enabled state, unplug a linear encoder cable for the rotator.

\medskip }
\end{minipage}
\\ \cdashline{2-2}


 & Expected Result \\
 & \begin{minipage}[t]{15cm}{\footnotesize
\smallskip
Confirm that a Linear Encoder Error event is created and the system
transitions to Fault state.

\medskip }
\end{minipage} \\ \cdashline{2-2}

 & Actual Result \\
 & \begin{minipage}[t]{15cm}{\footnotesize
\smallskip

\medskip }
\end{minipage} \\ \cdashline{2-2}

 & Status: \textbf{ Not Executed } \\ \hline

33 & Description \\
 & \begin{minipage}[t]{15cm}
{\footnotesize
\smallskip
Set the Following Error Threshold parameter to a very small value
(0.0001 deg or smaller) and send a PositionSet and then Move command.

\medskip }
\end{minipage}
\\ \cdashline{2-2}


 & Expected Result \\
 & \begin{minipage}[t]{15cm}{\footnotesize
\smallskip
Confirm that a Following Error event is created and the system
transitions to Fault state.

\medskip }
\end{minipage} \\ \cdashline{2-2}

 & Actual Result \\
 & \begin{minipage}[t]{15cm}{\footnotesize
\smallskip

\medskip }
\end{minipage} \\ \cdashline{2-2}

 & Status: \textbf{ Not Executed } \\ \hline

34 & Description \\
 & \begin{minipage}[t]{15cm}
{\footnotesize
\smallskip
Activate the positive software limit using a special control program.

\medskip }
\end{minipage}
\\ \cdashline{2-2}


 & Expected Result \\
 & \begin{minipage}[t]{15cm}{\footnotesize
\smallskip
Confirm that a Positive Limit Switch error message is created and the
system transitions to Fault state.

\medskip }
\end{minipage} \\ \cdashline{2-2}

 & Actual Result \\
 & \begin{minipage}[t]{15cm}{\footnotesize
\smallskip

\medskip }
\end{minipage} \\ \cdashline{2-2}

 & Status: \textbf{ Not Executed } \\ \hline

35 & Description \\
 & \begin{minipage}[t]{15cm}
{\footnotesize
\smallskip
Activate the negative software limit using a special control program.

\medskip }
\end{minipage}
\\ \cdashline{2-2}


 & Expected Result \\
 & \begin{minipage}[t]{15cm}{\footnotesize
\smallskip
Confirm that a Negative Limit Switch error message is created and the
system transitions to Fault State.

\medskip }
\end{minipage} \\ \cdashline{2-2}

 & Actual Result \\
 & \begin{minipage}[t]{15cm}{\footnotesize
\smallskip

\medskip }
\end{minipage} \\ \cdashline{2-2}

 & Status: \textbf{ Not Executed } \\ \hline

36 & Description \\
 & \begin{minipage}[t]{15cm}
{\footnotesize
\smallskip
Unplug the Ethercat cable between the control PC and the Copley XE2
drive.\\[2\baselineskip]

\medskip }
\end{minipage}
\\ \cdashline{2-2}


 & Expected Result \\
 & \begin{minipage}[t]{15cm}{\footnotesize
\smallskip
Confirm that an Ethercat Problem event is created and the system
transitions to Fault state.

\medskip }
\end{minipage} \\ \cdashline{2-2}

 & Actual Result \\
 & \begin{minipage}[t]{15cm}{\footnotesize
\smallskip

\medskip }
\end{minipage} \\ \cdashline{2-2}

 & Status: \textbf{ Not Executed } \\ \hline

37 & Description \\
 & \begin{minipage}[t]{15cm}
{\footnotesize
\smallskip
Perform a shutdown of the rotator.

\medskip }
\end{minipage}
\\ \cdashline{2-2}


 & Expected Result \\
 & \begin{minipage}[t]{15cm}{\footnotesize
\smallskip
Rotator is shutdown.

\medskip }
\end{minipage} \\ \cdashline{2-2}

 & Actual Result \\
 & \begin{minipage}[t]{15cm}{\footnotesize
\smallskip

\medskip }
\end{minipage} \\ \cdashline{2-2}

 & Status: \textbf{ Not Executed } \\ \hline

38 & Description \\
 & \begin{minipage}[t]{15cm}
{\footnotesize
\smallskip
\textbf{Section 5.0 of the attached Software Acceptance Test
Procedure}\\
\textbf{Rotator Telemetry Verification}\\[2\baselineskip]Using the EFD,
verify that the following telemetry streams were published at a rate of
20Hz.

\begin{itemize}
\tightlist
\item
  \textbf{rotator\_Application, Demand:} Commanded rotator position in
  degrees.
\item
  \textbf{rotator\_Application, Position:} Actual rotator position in
  degrees.
\item
  \textbf{rotator\_Application, Error:} Indication of a rotator
  following error.
\item
  \textbf{rotator\_Motors, Calibrated:} Encoder readings from each
  rotator motor scaled to degrees of the rotator.
\item
  \textbf{rotator\_Motors, Raw:} Encoder readings from each rotator
  motor in counts.
\item
  \textbf{rotator\_Electrical, CopleyStatusWordDrive:}
\end{itemize}

\medskip }
\end{minipage}
\\ \cdashline{2-2}


 & Expected Result \\
 & \begin{minipage}[t]{15cm}{\footnotesize
\smallskip
The defined telemetry streams were published to the EFD at 20Hz.

\medskip }
\end{minipage} \\ \cdashline{2-2}

 & Actual Result \\
 & \begin{minipage}[t]{15cm}{\footnotesize
\smallskip

\medskip }
\end{minipage} \\ \cdashline{2-2}

 & Status: \textbf{ Not Executed } \\ \hline

\end{longtable}

\paragraph{Test Case LVV-T1576 - Camera Rotator Hardware Functional Re-verification
 }\mbox{}\\

Open  \href{https://jira.lsstcorp.org/secure/Tests.jspa#/testCase/LVV-T1576}{\textit{ LVV-T1576 } }
test case in Jira.

The objective of this test case is to re-verify the functional
requirements of the camera rotator's hardware, after shipment from the
vendors facility to the Summit, as defined in \citeds{LTS-206}. This test case
will only exercise the functionality that was executed previously and
meets the following criteria:

\begin{itemize}
\tightlist
\item
  Only requires the camera rotator to be operable
\item
  Only requires the vendors EUI software and hardware via local control
\item
  Only requires a laser tracker
\item
  Does \textbf{NOT} require the camera rotator to be loaded with the
  camera simulated mass or actual camera hardware
\end{itemize}

The hardware functional requirements were previously verified during the
test campaign by the vendor at the vendors facility and accepted by LSST
during the Factory Acceptance Test review. The test procedure used
during the vendor's acceptance testing is the \emph{LSST
Hexapods-Rotator Acceptance Test Procedure} which is attached to this
test case. The test steps of this test case reference that document for
the details on how to perform the test in a similar way as was performed
previously and includes deviations to that document due to the
differences in the verification configuration and deviations to
requirements granted to the vendor by LSST.\\[2\baselineskip]The
\emph{LSST Hexapods-Rotator Acceptance Test Report} has also been
attached to this test case for additional information on how the tests
were performed. The section numbering in this document matches that of
the procedure.\\[2\baselineskip]See the attached \emph{LSST Rotator
Operator's Manual} for more information on how to operate the rotator.


\textbf{ Preconditions}:\\
{Prior to the execution of this test case to re-verify the Camera
Rotator hardware functional requirements, the following Summit tasks
must be completed:}

\begin{itemize}
\tightlist
\item
  The cables and cabinets have been checked~

  \begin{itemize}
  \tightlist
  \item
    \url{https://jira.lsstcorp.org/browse/SUMMIT-3231}
  \end{itemize}
\item
  Boxes for the hexapod/rotator have been transported to the 3rd level

  \begin{itemize}
  \tightlist
  \item
    \url{https://jira.lsstcorp.org/browse/SUMMIT-3230}
  \end{itemize}
\item
  Test fit Camera Hexapod with Offset

  \begin{itemize}
  \tightlist
  \item
    \url{https://jira.lsstcorp.org/browse/SUMMIT-3293}
  \end{itemize}
\item
  The Hexapod and Rotator have been installed on camera cart

  \begin{itemize}
  \tightlist
  \item
    \url{https://jira.lsstcorp.org/browse/SUMMIT-3224}
  \end{itemize}
\item
  The Camera hexapod/rotator has been connected to the electronics
  cabinets and the connections have been tested

  \begin{itemize}
  \tightlist
  \item
    \url{https://jira.lsstcorp.org/browse/SUMMIT-3294}
  \end{itemize}
\item
  The Offset has been installed on the Integrator

  \begin{itemize}
  \tightlist
  \item
    \url{https://jira.lsstcorp.org/browse/SUMMIT-3281}
  \end{itemize}
\item
  The setup for the laser tracker, current probes and data acquisition
  system has been completed

  \begin{itemize}
  \tightlist
  \item
    \url{https://jira.lsstcorp.org/browse/SUMMIT-3431}
  \end{itemize}
\end{itemize}


Execution status: {\bf Not Executed }

Final comment:\\


Detailed steps results:

\begin{longtable}{p{1cm}p{15cm}}
\hline
{Step} & Step Details\\ \hline
1 & Description \\
 & \begin{minipage}[t]{15cm}
{\footnotesize
\smallskip
\textbf{STARTING THE EUI}\\[2\baselineskip]Double click the Hexapod GUI
Viewer desktop icon on the computer.

\begin{itemize}
\tightlist
\item
  This can be done on the Dell Management PC or another computer on the
  same network
\end{itemize}

\medskip }
\end{minipage}
\\ \cdashline{2-2}


 & Expected Result \\
 & \begin{minipage}[t]{15cm}{\footnotesize
\smallskip
A prompt to enter the password is shown.

\medskip }
\end{minipage} \\ \cdashline{2-2}

 & Actual Result \\
 & \begin{minipage}[t]{15cm}{\footnotesize
\smallskip

\medskip }
\end{minipage} \\ \cdashline{2-2}

 & Status: \textbf{ Not Executed } \\ \hline

2 & Description \\
 & \begin{minipage}[t]{15cm}
{\footnotesize
\smallskip
Enter the password ``lsst-vnc''

\begin{itemize}
\tightlist
\item
  If the EUI isn't automatically up and running when the VNC opens,
  double click on the CAM\_Hex\_eGUI or M2\_Hex\_eGUI icon on the VNC
  viewer
\end{itemize}

\medskip }
\end{minipage}
\\ \cdashline{2-2}


 & Expected Result \\
 & \begin{minipage}[t]{15cm}{\footnotesize
\smallskip
The EUI is in the Offline State/PublishOnly substate and is able to
publish through SAL but cannot receive commands.

\medskip }
\end{minipage} \\ \cdashline{2-2}

 & Actual Result \\
 & \begin{minipage}[t]{15cm}{\footnotesize
\smallskip

\medskip }
\end{minipage} \\ \cdashline{2-2}

 & Status: \textbf{ Not Executed } \\ \hline

3 & Description \\
 & \begin{minipage}[t]{15cm}
{\footnotesize
\smallskip
\textbf{OFFLINESTATE/AVAILABLESTATE}\\
On the Main tab, select the ``Offline SubState Cmd'' field in the
Commands to Send section, set the Offline SubState Triggers to ``System
Ready'' and click on the Send Command button.\\
\includegraphics[width=1.79167in]{jira_imgs/1005.png}

\medskip }
\end{minipage}
\\ \cdashline{2-2}


 & Expected Result \\
 & \begin{minipage}[t]{15cm}{\footnotesize
\smallskip
The system transitions from the OfflineState/PublishOnly substate to the
OfflineState/AvailableState
substate.\\[2\baselineskip]\includegraphics[width=3.12500in]{jira_imgs/1007.png}

\medskip }
\end{minipage} \\ \cdashline{2-2}

 & Actual Result \\
 & \begin{minipage}[t]{15cm}{\footnotesize
\smallskip

\medskip }
\end{minipage} \\ \cdashline{2-2}

 & Status: \textbf{ Not Executed } \\ \hline

4 & Description \\
 & \begin{minipage}[t]{15cm}
{\footnotesize
\smallskip
\textbf{OFFLINESTATE -\textgreater{} STANDBYSTATE}\\
Click on the State Command field in the Commands to Send section.\\
\includegraphics[width=1.79167in]{jira_imgs/1030.png}

\medskip }
\end{minipage}
\\ \cdashline{2-2}


 & Expected Result \\
 & \begin{minipage}[t]{15cm}{\footnotesize
\smallskip
The system transitions into the StandbyState and the primary state
display box at the top of the Main tab says Standby State.\\
\includegraphics[width=4.68750in]{jira_imgs/1018.png}

\medskip }
\end{minipage} \\ \cdashline{2-2}

 & Actual Result \\
 & \begin{minipage}[t]{15cm}{\footnotesize
\smallskip

\medskip }
\end{minipage} \\ \cdashline{2-2}

 & Status: \textbf{ Not Executed } \\ \hline

5 & Description \\
 & \begin{minipage}[t]{15cm}
{\footnotesize
\smallskip
\textbf{STANDBYSTATE -\textgreater{} DISABLEDSTATE}\\
From the StandbyState, send a start command.

\medskip }
\end{minipage}
\\ \cdashline{2-2}


 & Expected Result \\
 & \begin{minipage}[t]{15cm}{\footnotesize
\smallskip
The system transitions into DisabledState and the current configuration
parameters are maintained from the default parameters or from the
previous DDS start command.~\\
\includegraphics[width=3.12500in]{jira_imgs/1019.png}\\
If the configuration file is invalid or out of range, the system will
transition into a Fault State

\medskip }
\end{minipage} \\ \cdashline{2-2}

 & Actual Result \\
 & \begin{minipage}[t]{15cm}{\footnotesize
\smallskip

\medskip }
\end{minipage} \\ \cdashline{2-2}

 & Status: \textbf{ Not Executed } \\ \hline

6 & Description \\
 & \begin{minipage}[t]{15cm}
{\footnotesize
\smallskip
\textbf{DISABLEDSTATE -\textgreater{} ENABLEDSTATE}\\
From the DisabledState, send an Enable state.

\medskip }
\end{minipage}
\\ \cdashline{2-2}


 & Expected Result \\
 & \begin{minipage}[t]{15cm}{\footnotesize
\smallskip
The system transitions into the EnabledState/Stationary substate, the
motor drives are enabled, and motion can be commanded.\\
\includegraphics[width=3.12500in]{jira_imgs/1020.png}\\

\medskip }
\end{minipage} \\ \cdashline{2-2}

 & Actual Result \\
 & \begin{minipage}[t]{15cm}{\footnotesize
\smallskip

\medskip }
\end{minipage} \\ \cdashline{2-2}

 & Status: \textbf{ Not Executed } \\ \hline

7 & Description \\
 & \begin{minipage}[t]{15cm}
{\footnotesize
\smallskip
\textbf{FAULTSTATE}\\
If a Fault occurs in any of the other states, the system will
automatically transition to the Fault State. While in the Fault state,
send a clearError command.\\
Note: If the fault that occurs goes through the interlock system, reset
the safety relay switch and send a clearError command.

\medskip }
\end{minipage}
\\ \cdashline{2-2}


 & Expected Result \\
 & \begin{minipage}[t]{15cm}{\footnotesize
\smallskip
The system transitions back to the OfflineState/PublishOnly substate.
(Go back to Step 3)\\
\includegraphics[width=3.12500in]{jira_imgs/1021.png}

\medskip }
\end{minipage} \\ \cdashline{2-2}

 & Actual Result \\
 & \begin{minipage}[t]{15cm}{\footnotesize
\smallskip

\medskip }
\end{minipage} \\ \cdashline{2-2}

 & Status: \textbf{ Not Executed } \\ \hline

8 & Description \\
 & \begin{minipage}[t]{15cm}
{\footnotesize
\smallskip
Follow Section 3.4.4 of the LSST Hexapods-Rotator Acceptance Test
Procedure, Sheet 47.

\medskip }
\end{minipage}
\\ \cdashline{2-2}

 & Test Data \\
 & \begin{minipage}[t]{15cm}{\footnotesize
\smallskip
\textbf{Deviation:} After verifying that the rotator can move through
it's operational range (+/- 90 deg) without triggering any limits,
adjust the software limit to in between the values for the Limit Switch
and End Stop as defined in the table below for SN 02 (taken from
vendor's Acceptance Test Report). Move the rotator to trip the positive
and negative limit switchs.\\
\includegraphics[width=1.79167in]{jira_imgs/984.png}\\
Do \textbf{NOT} verify the rotator's End Stop hardpoints.

\medskip }
\end{minipage} \\ \cdashline{2-2}

 & Expected Result \\
 & \begin{minipage}[t]{15cm}{\footnotesize
\smallskip
\begin{enumerate}
\tightlist
\item
  {The software limit prevents the camera rotator from rotating beyond
  +/-90deg.}
\item
  The limit switch stops the rotator before +/- 92 deg.
\end{enumerate}

\medskip }
\end{minipage} \\ \cdashline{2-2}

 & Actual Result \\
 & \begin{minipage}[t]{15cm}{\footnotesize
\smallskip

\medskip }
\end{minipage} \\ \cdashline{2-2}

 & Status: \textbf{ Not Executed } \\ \hline

9 & Description \\
 & \begin{minipage}[t]{15cm}
{\footnotesize
\smallskip
Follow Section 3.4.1 of the LSST Hexapods-Rotator Acceptance Test
Procedure, Sheet 46.

\medskip }
\end{minipage}
\\ \cdashline{2-2}


 & Expected Result \\
 & \begin{minipage}[t]{15cm}{\footnotesize
\smallskip
The axis of rotation is visually confirmed to be along the z-axis and is
consistent to the results of the initial tests conducted by the vendor
as seen in the LSST Hexapods-Rotator Acceptance Test Report, Sheet 48.

\medskip }
\end{minipage} \\ \cdashline{2-2}

 & Actual Result \\
 & \begin{minipage}[t]{15cm}{\footnotesize
\smallskip

\medskip }
\end{minipage} \\ \cdashline{2-2}

 & Status: \textbf{ Not Executed } \\ \hline

10 & Description \\
 & \begin{minipage}[t]{15cm}
{\footnotesize
\smallskip
Follow Section 3.4.2 of the LSST Hexapods-Rotator Acceptance Test
Procedure, Sheet 46 using a laser tracker.

\medskip }
\end{minipage}
\\ \cdashline{2-2}

 & Test Data \\
 & \begin{minipage}[t]{15cm}{\footnotesize
\smallskip
\textbf{Deviation:} Instead of using 5 different elevation angles, the
test will only be conducted at a 0deg elevation angle

\medskip }
\end{minipage} \\ \cdashline{2-2}

 & Expected Result \\
 & \begin{minipage}[t]{15cm}{\footnotesize
\smallskip
The Camera Rotator is tested every 30 degrees across the entire range of
motion and the maximum angle error is found to be less than 0.009
degrees.

\medskip }
\end{minipage} \\ \cdashline{2-2}

 & Actual Result \\
 & \begin{minipage}[t]{15cm}{\footnotesize
\smallskip

\medskip }
\end{minipage} \\ \cdashline{2-2}

 & Status: \textbf{ Not Executed } \\ \hline

11 & Description \\
 & \begin{minipage}[t]{15cm}
{\footnotesize
\smallskip
Follow Section 3.4.5.1 of the LSST Hexapods-Rotator Acceptance Test
Procedure, Sheet 48.

\medskip }
\end{minipage}
\\ \cdashline{2-2}


 & Expected Result \\
 & \begin{minipage}[t]{15cm}{\footnotesize
\smallskip
The Camera Rotator is able to reach the required velocity of 3.5 deg/s
as verified before per the LSST Hexapods-Rotator Acceptance Test Report,
Sheet 50-52

\medskip }
\end{minipage} \\ \cdashline{2-2}

 & Actual Result \\
 & \begin{minipage}[t]{15cm}{\footnotesize
\smallskip

\medskip }
\end{minipage} \\ \cdashline{2-2}

 & Status: \textbf{ Not Executed } \\ \hline

12 & Description \\
 & \begin{minipage}[t]{15cm}
{\footnotesize
\smallskip
Follow Section 3.4.5.2 of the LSST Hexapods-Rotator Acceptance Test
Procedure, Sheet 49.

\medskip }
\end{minipage}
\\ \cdashline{2-2}


 & Expected Result \\
 & \begin{minipage}[t]{15cm}{\footnotesize
\smallskip
The Camera Rotator is able to reach the required acceleration of 1.0
degrees/sec\^{}2 as verified before per the LSST Hexapods\_Rotator
Acceptance Test Report, Sheet 52.

\medskip }
\end{minipage} \\ \cdashline{2-2}

 & Actual Result \\
 & \begin{minipage}[t]{15cm}{\footnotesize
\smallskip

\medskip }
\end{minipage} \\ \cdashline{2-2}

 & Status: \textbf{ Not Executed } \\ \hline

13 & Description \\
 & \begin{minipage}[t]{15cm}
{\footnotesize
\smallskip
Follow Section 3.4.5.3 of the LSST Hexapods-Rotator Acceptance Test
Procedure, Sheet 49.

\medskip }
\end{minipage}
\\ \cdashline{2-2}

 & Test Data \\
 & \begin{minipage}[t]{15cm}{\footnotesize
\smallskip
\textbf{Deviation:~}Steps 7, 8 and 9 (Section 3.4.5.3, 3.4.5.4 and
3.4.5.5) can be tested simultaneously with the same data.

\medskip }
\end{minipage} \\ \cdashline{2-2}

 & Expected Result \\
 & \begin{minipage}[t]{15cm}{\footnotesize
\smallskip
The initial result of the test (as seen in LSST Hexapods\_Rotator
Acceptance Test Report, Sheet 52-54) found that the requirement was not
met, but was accepted per deviation request
\url{https://jira.lsstcorp.org/browse/LVV-7218}\emph{. The
re-verification only needs to meet the values approved in the
deviation.\\
}

\medskip }
\end{minipage} \\ \cdashline{2-2}

 & Actual Result \\
 & \begin{minipage}[t]{15cm}{\footnotesize
\smallskip

\medskip }
\end{minipage} \\ \cdashline{2-2}

 & Status: \textbf{ Not Executed } \\ \hline

14 & Description \\
 & \begin{minipage}[t]{15cm}
{\footnotesize
\smallskip
Follow Section 3.4.5.4 of the LSST Hexapods-Rotator Acceptance Test
Procedure, Sheet 49.

\medskip }
\end{minipage}
\\ \cdashline{2-2}

 & Test Data \\
 & \begin{minipage}[t]{15cm}{\footnotesize
\smallskip
\textbf{Deviation:~}Perform this step at a 0 degree elevation angle
instead of the 90 degree angle listed in the procedure and use the same
data used in Step 7.~

\medskip }
\end{minipage} \\ \cdashline{2-2}

 & Expected Result \\
 & \begin{minipage}[t]{15cm}{\footnotesize
\smallskip
The maximum radial displacement is measured to be under 50 microns as
previously verified per LSST Hexapods\_Rotator Acceptance Test Report,
Sheet 54.

\medskip }
\end{minipage} \\ \cdashline{2-2}

 & Actual Result \\
 & \begin{minipage}[t]{15cm}{\footnotesize
\smallskip

\medskip }
\end{minipage} \\ \cdashline{2-2}

 & Status: \textbf{ Not Executed } \\ \hline

15 & Description \\
 & \begin{minipage}[t]{15cm}
{\footnotesize
\smallskip
Follow Section 3.4.5.5 of the LSST Hexapods-Rotator Acceptance Test
Procedure, Sheet 50.

\medskip }
\end{minipage}
\\ \cdashline{2-2}

 & Test Data \\
 & \begin{minipage}[t]{15cm}{\footnotesize
\smallskip
\textbf{Deviation:~}Perform this step at a 0 degree elevation angle
instead of the 90 degree angle listed in the procedure and use the same
data used in Step 7.~

\medskip }
\end{minipage} \\ \cdashline{2-2}

 & Expected Result \\
 & \begin{minipage}[t]{15cm}{\footnotesize
\smallskip
{The maximum radial displacement is measured to be under 100 microns as
previously verified per LSST Hexapods\_Rotator Acceptance Test Report,
Sheet 55. }

\medskip }
\end{minipage} \\ \cdashline{2-2}

 & Actual Result \\
 & \begin{minipage}[t]{15cm}{\footnotesize
\smallskip

\medskip }
\end{minipage} \\ \cdashline{2-2}

 & Status: \textbf{ Not Executed } \\ \hline

16 & Description \\
 & \begin{minipage}[t]{15cm}
{\footnotesize
\smallskip
Follow Section 3.4.5.6 of the LSST Hexapods-Rotator Acceptance Test
Procedure, Sheet 50.

\medskip }
\end{minipage}
\\ \cdashline{2-2}

 & Test Data \\
 & \begin{minipage}[t]{15cm}{\footnotesize
\smallskip
\textbf{Deviation:} This will be done with a laser tracker knowing that
we will not be able to measure as low as 2 microns.~

\medskip }
\end{minipage} \\ \cdashline{2-2}

 & Expected Result \\
 & \begin{minipage}[t]{15cm}{\footnotesize
\smallskip
{Since the error cannot be measured up to 2 microns, the camera
rotator's accuracy is found to be as good as the laser tracker's
accuracy.}

\medskip }
\end{minipage} \\ \cdashline{2-2}

 & Actual Result \\
 & \begin{minipage}[t]{15cm}{\footnotesize
\smallskip

\medskip }
\end{minipage} \\ \cdashline{2-2}

 & Status: \textbf{ Not Executed } \\ \hline

17 & Description \\
 & \begin{minipage}[t]{15cm}
{\footnotesize
\smallskip
To verify 3.4.6.1, follow Section 3.4.6.2 of the LSST Hexapods-Rotator
Acceptance Test Procedure, Sheet 50-52.

\medskip }
\end{minipage}
\\ \cdashline{2-2}

 & Test Data \\
 & \begin{minipage}[t]{15cm}{\footnotesize
\smallskip
\textbf{Deviation:} Steps 10 and 11 (Section 3.4.6.1 and 3.4.6.2) will
be tested simultaneously.\\
We will be testing all speeds from 0.005 to 0.068deg/s just like in LSST
Hexapods-Rotator Acceptance Test Report Sheets 57-64. However, we will
only be using the encoder (no laser interferometer) and one elevation
angle (0 Deg El) to verify.

\medskip }
\end{minipage} \\ \cdashline{2-2}

 & Expected Result \\
 & \begin{minipage}[t]{15cm}{\footnotesize
\smallskip
\begin{enumerate}
\tightlist
\item
  The SN02 Rotator is found to be able to reach all of the testing
  speeds from 0.005 to 0.068deg/s with a zero degree elevation angle.~
\item
  The SN02 Rotator's Tracking Accuracy is recorded for all tracking
  velocities from 0.05deg/s to 0.068deg/s with a zero degree elevation
  angle and is found to have a position error equal or better than 0.1
  arcs seconds RMS.
\end{enumerate}

\medskip }
\end{minipage} \\ \cdashline{2-2}

 & Actual Result \\
 & \begin{minipage}[t]{15cm}{\footnotesize
\smallskip

\medskip }
\end{minipage} \\ \cdashline{2-2}

 & Status: \textbf{ Not Executed } \\ \hline

18 & Description \\
 & \begin{minipage}[t]{15cm}
{\footnotesize
\smallskip
Follow Section 3.4.6.6 of the LSST Hexapods-Rotator Acceptance Test
Procedure, Sheet 53

\medskip }
\end{minipage}
\\ \cdashline{2-2}

 & Test Data \\
 & \begin{minipage}[t]{15cm}{\footnotesize
\smallskip
\textbf{Deviation:~}This will only be conducted at 0 deg elevation angle
since it is on the camera cart.

\medskip }
\end{minipage} \\ \cdashline{2-2}

 & Expected Result \\
 & \begin{minipage}[t]{15cm}{\footnotesize
\smallskip
By checking the current of the system, the heat dissipation for the
rotator is verified to be less than 40W.

\medskip }
\end{minipage} \\ \cdashline{2-2}

 & Actual Result \\
 & \begin{minipage}[t]{15cm}{\footnotesize
\smallskip

\medskip }
\end{minipage} \\ \cdashline{2-2}

 & Status: \textbf{ Not Executed } \\ \hline

19 & Description \\
 & \begin{minipage}[t]{15cm}
{\footnotesize
\smallskip
Follow Section 3.4.9 of the LSST Hexapods-Rotator Acceptance Test
Procedure, Sheet 54.

\medskip }
\end{minipage}
\\ \cdashline{2-2}


 & Expected Result \\
 & \begin{minipage}[t]{15cm}{\footnotesize
\smallskip
The locking pin is demonstrated to be able to engage at 15 deg intervals
throughout the entire rotator range while the camera hexapod/rotator is
installed on the camera cart.

\medskip }
\end{minipage} \\ \cdashline{2-2}

 & Actual Result \\
 & \begin{minipage}[t]{15cm}{\footnotesize
\smallskip

\medskip }
\end{minipage} \\ \cdashline{2-2}

 & Status: \textbf{ Not Executed } \\ \hline

\end{longtable}


\input{appendix.tex}
\end{document}
